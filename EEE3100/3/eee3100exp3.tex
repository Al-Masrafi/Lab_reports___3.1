\documentclass[a4paper,12pt]{article}

\usepackage{graphicx} % Required for inserting images
\usepackage{amsmath,amssymb,amsfonts}
\usepackage{subcaption}

% -----------------------
% Package Imports
% -----------------------

% Set page margins
\usepackage[a4paper, top=1in, bottom=0.8in, left=1.1in, right=0.8in]{geometry}

% Use Times New Roman font
\usepackage{times}

% Add page numbering
\pagestyle{plain}
\usepackage{multirow}
\usepackage{float}

% Enable code listings
\usepackage{listings}
\usepackage{xcolor} % For customizing code colors

% Use Courier New for code (requires 'courier' package)
\usepackage{courier}

% Define style for code listings using Courier New and 10pt, no line numbers
\lstdefinestyle{courier10}{
	language=Matlab,
	basicstyle=\fontfamily{pcr}\selectfont\fontsize{10}{12}\selectfont,
	keywordstyle=\color{blue},
	commentstyle=\color{gray},
	stringstyle=\color{red},
	frame=single,
	backgroundcolor=\color{white!10},
	breaklines=true,
	captionpos=b,
	tabsize=4,
	showstringspaces=false,
	numbers=none  % No line numbers
}

\setlength{\parindent}{0pt}
\usepackage{sectsty}
\sectionfont{\fontsize{12}{15}\selectfont}

\begin{document}
	
	\textbf{Experiment No: 3} \\
	\textbf{Name of the Experiment:} Hands-On Projects with Arduino Microcontroller \\
	{\fontsize{14pt}{16pt}\selectfont\textbf{Name of the Experiment:}} Hands-On Projects with Arduino Microcontroller \\
	
	\textbf{Objective:}
	
	
	\begin{itemize}
		\item To learn how to interface basic hardware with the Arduino microcontroller.
		\item To develop simple code for hardware control using Arduino IDE.
	\end{itemize}
	
	
	
	{\fontsize{14pt}{0pt}\selectfont\textbf{Project No.: 01}}  
	
	\textbf{Project Name:}
    \section*{Overview:}
    \section*{Required Apparatus:}
	\section*{Circuit Diagram:}
	\section*{Experimental Setup:}
	\newpage
	\vspace{0.5cm}
	\textbf{Code:}
	
	\begin{lstlisting}[style=courier10, caption={Blink LED Code}]
		// Motor A connections
		int enA = 9;
		int in1 = 8;
		int in2 = 7;
		// Motor B connections
		int enB = 3;
		int in3 = 5;
		int in4 = 4;
		bool a = true;
		void setup() {
			// Set all the motor control pins to outputs
			pinMode(enA, OUTPUT);
			pinMode(enB, OUTPUT);
			pinMode(in1, OUTPUT);
			pinMode(in2, OUTPUT);
			pinMode(in3, OUTPUT);
			pinMode(in4, OUTPUT);
			pinMode(12, INPUT);
			// Turn off motors - Initial state
			digitalWrite(in1, LOW);
			digitalWrite(in2, LOW);
			digitalWrite(in3, LOW);
			digitalWrite(in4, LOW);
		}
		
		void loop() {
			if (a == true) {
				analogWrite(enA, 255);
				digitalWrite(in1, HIGH);
				digitalWrite(in2, LOW);
				delay(5000);
				
				a = false;
				digitalWrite(in1, LOW);
				digitalWrite(in2, LOW);
			}
			
			if (digitalRead(12) == HIGH) {
				analogWrite(enB, 255);
				// Motor 2 run 5 sec
				digitalWrite(in3, HIGH);
				digitalWrite(in4, LOW);
				delay(5000);
				digitalWrite(in3, LOW);
				digitalWrite(in4, LOW);
				delay(1000);
				// Now run both motors for 10 seconds
				analogWrite(enA, 255);
				analogWrite(enB, 255);
				
				digitalWrite(in1, HIGH);
				digitalWrite(in2, LOW);
				digitalWrite(in3, HIGH);
				digitalWrite(in4, LOW);
					\end{lstlisting}
					\newpage
					\begin{lstlisting}[style=courier10, caption={Blink LED Code}]
				delay(10000);
				digitalWrite(in1, LOW);
				digitalWrite(in2, LOW);
				digitalWrite(in3, LOW);
				digitalWrite(in4, LOW);
				delay(1000);
				a = true;
			}
		}
		// This function lets you control spinning direction of motors
		void directionControl() {
			// Set motors to maximum speed
			// For PWM maximum possible values are 0 to 255
			analogWrite(enA, 255);
			analogWrite(enB, 255);
			
			// Turn on motor A & B
			digitalWrite(in1, HIGH);
			digitalWrite(in2, LOW);
			digitalWrite(in3, HIGH);
			digitalWrite(in4, LOW);
			delay(2000);
			
			// Now change motor directions
			digitalWrite(in1, LOW);
			digitalWrite(in2, HIGH);
			digitalWrite(in3, LOW);
			digitalWrite(in4, HIGH);
			delay(2000);
			
			// Turn off motors
			digitalWrite(in1, LOW);
			digitalWrite(in2, LOW);
			digitalWrite(in3, LOW);
			digitalWrite(in4, LOW);
		}
		// This function lets you control speed of the motors
		void speedControl() {
			// Turn on motors
			digitalWrite(in1, LOW);
			digitalWrite(in2, HIGH);
			digitalWrite(in3, LOW);
			digitalWrite(in4, HIGH);
			
			// Accelerate from zero to maximum speed
			for (int i = 0; i < 256; i++) {
				analogWrite(enA, i);
				analogWrite(enB, i);
				delay(20);
			}
			
			// Decelerate from maximum speed to zero
			for (int i = 255; i >= 0; --i) {
				analogWrite(enA, i);
				analogWrite(enB, i);
				delay(20);
			}
				\end{lstlisting}
				
				
					\begin{lstlisting}[style=courier10, caption={Blink LED Code}]
			// Now turn off motors
			digitalWrite(in1, LOW);
			digitalWrite(in2, LOW);
			digitalWrite(in3, LOW);
			digitalWrite(in4, LOW);
		}
		
		void maruf() {
			analogWrite(enA, 255);
			analogWrite(enB, 255);
			for (int i = 1; i < 10; i++) {
				digitalWrite(in1, HIGH);
				digitalWrite(in2, LOW);
				digitalWrite(in3, HIGH);
				digitalWrite(in4, LOW);
				delay(200);
				
				digitalWrite(in1, LOW);
				digitalWrite(in2, HIGH);
				digitalWrite(in3, LOW);
				digitalWrite(in4, HIGH);
				delay(200);
			}
		}
		
	\end{lstlisting}
	
	\section*{Discussions:}
	
	
	
	
	
	
	
	
	
	
	
	
\end{document}
