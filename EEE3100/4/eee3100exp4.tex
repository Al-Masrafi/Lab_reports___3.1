\documentclass[a4paper,12pt]{article}

\usepackage{graphicx} % Required for inserting images
\usepackage{amsmath,amssymb,amsfonts}
\usepackage{subcaption}

% -----------------------
% Package Imports
% -----------------------

% Set page margins
\usepackage[a4paper, top=1in, bottom=0.8in, left=1.1in, right=0.8in]{geometry}

% Use Times New Roman font
\usepackage{times}
\usepackage{multicol}

% Add page numbering
\pagestyle{plain}
\usepackage{multirow}
\usepackage{float}

% Enable code listings
\usepackage{listings}
\usepackage{xcolor} % For customizing code colors

% Use Courier New for code (requires 'courier' package)
\usepackage{courier}

% Define style for code listings using Courier New and 10pt, no line numbers
\lstdefinestyle{courier10}{
	language=Matlab,
	basicstyle=\fontfamily{pcr}\selectfont\fontsize{10}{12}\selectfont,
	keywordstyle=\color{blue},
	commentstyle=\color{gray},
	stringstyle=\color{red},
	frame=single,
	backgroundcolor=\color{white!10},
	breaklines=true,
	captionpos=b,
	tabsize=4,
	showstringspaces=false,
	numbers=none  % No line numbers
}

\setlength{\parindent}{0pt}
\usepackage{sectsty}
\sectionfont{\fontsize{12}{15}\selectfont}

\begin{document}
	
	\textbf{Experiment No: }4\\
	\textbf{ Experiment Name:} Control and Analyze the Operation of DC Motors Using Arduino and L293D
	Motor Driver IC. \\

	\textbf{Objective:}
	
	
	\begin{itemize}
		\item To learn how to interface basic hardware with the Arduino microcontroller.
		\item To develop simple code for hardware control using Arduino IDE.
	\end{itemize}
	
	
	

	\section*{Theory:}
	The implementation of robotic systems often requires precise control of DC motors, which are the primary actuators used to produce movement. However, microcontrollers like the Arduino cannot directly drive these motors due to current limitations. The \textbf{L293D Motor Driver IC} serves as a crucial interface between the low-power control circuit and the high-power motors. It acts as a bridge that allows a microcontroller to control the direction and speed of motors safely and efficiently.
	
	\begin{figure}[H]
		\centering
			\includegraphics[width=0.5\linewidth]{"Images/2"}
		\caption*{\textbf{Figure 4.1:} L293D Motor Driver IC}
	\end{figure}
	
	\section*{Overview of L293D Motor Driver IC:}
	The L293D is a dual H-Bridge motor driver IC, which means it can drive two DC motors independently in both forward and reverse directions. It operates on a 4.5V to 36V supply range for the motor and has separate logic and motor supply pins. The L293D can supply a maximum current of 600mA per channel, which is suitable for most small DC motors used in educational and hobby-grade robotic systems.
	
	Internally, each H-Bridge is composed of transistor pairs arranged in such a way that changing the input logic levels can reverse the polarity of the motor voltage. This capability is essential for robotic systems that need bidirectional control.
	
	\section*{Pin Configuration and Functionality}
	
	The IC consists of 16 pins with the following key connections:
	
	\begin{enumerate}
		\item \textbf{Input Pins (IN1, IN2, IN3, IN4):} Receive logic signals from the microcontroller.
		\item \textbf{Output Pins (OUT1, OUT2, OUT3, OUT4):} Connected to the motor terminals.
		\item \textbf{Enable Pins (EN1, EN2):} Control whether each motor is enabled. These must be HIGH to allow motor operation.
		\item \textbf{VCC1 (Logic Voltage):} Typically 5V, powers the internal logic circuit.
		\item \textbf{VCC2 (Motor Voltage):} Powers the motors (up to 36V).
		\item \textbf{GND:} Common ground for both logic and motor power.
	\end{enumerate}
	
	\noindent The basic truth table for motor control using one H-Bridge of the L293D is shown below:
	
	\begin{center}
		\begin{tabular}{|c|c|c|}
			\hline
			IN1 & IN2 & Motor Direction \\
			\hline
			1 & 0 & Forward \\
			0 & 1 & Reverse \\
			1 & 1 & Brake \\
			0 & 0 & Stop \\
			\hline
		\end{tabular}
	\end{center}
	
	\section*{Working Principle}
	
	When a HIGH signal is sent to an input pin, the corresponding output pin goes HIGH (assuming the enable pin is also HIGH). By adjusting the combination of HIGH and LOW signals at the input pins, one can control the rotation direction of the motors. This technique is particularly useful for differential drive robotic systems, where varying the direction and speed of two motors allows for forward movement, reverse movement, and turning.
	
	\section*{Integration with Arduino for Robotic Control}
	
	In a robotic system, the L293D is connected between the Arduino and the DC motors. The Arduino sends digital HIGH/LOW signals to the input pins of the L293D. The enable pins can be tied to HIGH for basic control or connected to PWM outputs of the Arduino to achieve speed control via pulse-width modulation (PWM).
	
	\noindent For example, to rotate a robot forward:
	\begin{enumerate}
		\item Left Motor: IN1 = HIGH, IN2 = LOW
		\item Right Motor: IN3 = HIGH, IN4 = LOW
	\end{enumerate}
	
	\noindent To turn left:
	\begin{enumerate}
		\item Left Motor: IN1 = LOW, IN2 = HIGH
		\item Right Motor: IN3 = HIGH, IN4 = LOW
	\end{enumerate}
	
	\noindent The robotic platform can thus navigate through the environment by altering motor signals dynamically based on sensor input or programmed behavior.
	
		\section*{Required Apparatus:}
	\begin{table}[H]
		\centering
		\caption{Components list for the experimental setup}
		\begin{tabular}{|c|c|c|c|}
			\hline
			\textbf{\begin{tabular}[c]{@{}c@{}}SI \\ No.\end{tabular}} & \textbf{Components}                                                                   & \textbf{Specifications}                                                                     & \textbf{Quantity} \\ \hline
			1                                                          & Micro-controller Board                                                                            & \begin{tabular}[c]{@{}c@{}}Arduino UNO\end{tabular} & 1                 \\ \hline
			2                                                          & Motor Driver                                                                             & \begin{tabular}[c]{@{}c@{}}L293D Motor Driver IC   \end{tabular} & 1                 \\ \hline
			3                                                          & Power Supply                                                                             & \begin{tabular}[c]{@{}c@{}} 15V variable DC \end{tabular}           & 1                 \\ \hline
			4                                                          &Push Button                                                                                   &                                                                                             & 1                 \\ \hline
			5                                                          & Connecting Wires                                                                         &  \begin{tabular}[c]{@{}c@{}}Male to Male Connector\end{tabular}                                                                                    & 1                 \\ \hline
			
		\end{tabular}
	\end{table}
	
	
	\section*{Circuit Diagram:}
		\begin{figure}[H]
		\centering
		\includegraphics[width=1\linewidth, height=.3\textheight]{"Images/1"}
		\caption{Schematic Diagram}
	\end{figure}
	
	\section*{Experimental Setup:}
		\begin{figure}[H]
		\centering
		\includegraphics[angle=270,width=0.81 \linewidth]{"Images/3"}
		\caption{Experimental Setup}
	\end{figure}
	\newpage
	\vspace{0.5cm}
	\textbf{Code:}
	
	\begin{lstlisting}[style=courier10, caption={Basic Motor operation Code}]
		// Motor A connections
		int enA = 9;
		int in1 = 8;
		int in2 = 7;
		// Motor B connections
		int enB = 3;
		int in3 = 5;
		int in4 = 4;
		bool a = true;
		void setup() {
			// Set all the motor control pins to outputs
			pinMode(enA, OUTPUT);
			pinMode(enB, OUTPUT);
			pinMode(in1, OUTPUT);
			pinMode(in2, OUTPUT);
			pinMode(in3, OUTPUT);
			pinMode(in4, OUTPUT);
			pinMode(12, INPUT);
			// Turn off motors - Initial state
			digitalWrite(in1, LOW);
			digitalWrite(in2, LOW);
			digitalWrite(in3, LOW);
			digitalWrite(in4, LOW);
		}
		void loop() {
				directionControl();
				delay(1000);
				speedControl();
				delay(1000);
			}
		// This function lets you control spinning direction of motors
		void directionControl() {
			// Set motors to maximum speed
			// For PWM maximum possible values are 0 to 255
			analogWrite(enA, 255);
			analogWrite(enB, 255);
			
			// Turn on motor A & B
			digitalWrite(in1, HIGH);
			digitalWrite(in2, LOW);
			digitalWrite(in3, HIGH);
			digitalWrite(in4, LOW);
			delay(2000);
			
			// Now change motor directions
			digitalWrite(in1, LOW);
			digitalWrite(in2, HIGH);
			digitalWrite(in3, LOW);
			digitalWrite(in4, HIGH);
			delay(2000);
			// Turn off motors
			digitalWrite(in1, LOW);
			digitalWrite(in2, LOW);
			digitalWrite(in3, LOW);
			digitalWrite(in4, LOW);
		}
		\end{lstlisting}
		\begin{lstlisting}[style=courier10, caption={Basic Motor operation Code}]
			
		
		// This function lets you control speed of the motors
		void speedControl() {
			// Turn on motors
			digitalWrite(in1, LOW);
			digitalWrite(in2, HIGH);
			digitalWrite(in3, LOW);
			digitalWrite(in4, HIGH);
			
			// Accelerate from zero to maximum speed
			for (int i = 0; i < 256; i++) {
				analogWrite(enA, i);
				analogWrite(enB, i);
				delay(20);
			}
			
			// Decelerate from maximum speed to zero
			for (int i = 255; i >= 0; --i) {
				analogWrite(enA, i);
				analogWrite(enB, i);
				delay(20);
			}
			// Now turn off motors
			digitalWrite(in1, LOW);
			digitalWrite(in2, LOW);
			digitalWrite(in3, LOW);
			digitalWrite(in4, LOW);
		}
		
	
	\end{lstlisting}
	
	
	
	
	
	
	\section*{Experimental Procedure:}
		For the basic (main) experiment, the following steps were executed:
	\begin{enumerate}
		\item Connected the L293D motor driver IC to the Arduino UNO as per the circuit diagram.
		\item Connected Motor A to pins IN1 (8) and IN2 (7), and Motor B to pins IN3 (5) and IN4 (4) of the Arduino.
		\item Connected the enable pins ENA (9) and ENB (3) to the PWM-capable pins of Arduino for speed control.
		\item Uploaded the Arduino code to the board using the Arduino IDE.
		\item In the \texttt{setup()} function, set all motor-related pins as \texttt{OUTPUT} and turned motors off initially.
		\item In the \texttt{loop()} function, first executed the \texttt{directionControl()} function to rotate both motors in one direction and then reverse.
		\item Used the \texttt{speedControl()} function to gradually increase and decrease motor speed using PWM signals.
		\item Observed motor behavior during directional and speed control phases.
	\end{enumerate}
	
	\newpage
	\begin{center}
		\textbf{Hands-on Learning Projects for Mastery of DC Motor Control with Arduino and L293D}
	\end{center}
	
\textbf{Task No.:} 01 \\
\textbf{	Task Name:} Designing Timed Sequential Control of Dual Motors in a Repetitive Cycle\\
	
		\textbf{Objective:}
	
	
	\begin{itemize}
		\item To implement a time-based motor control system where Motor 1 runs for a specified
		duration, followed by Motor 2, and then both motors run together, forming a repeatable
		operational cycle
	\end{itemize}
	
	

	
\textbf{Code:}
\begin{lstlisting}[style=courier10, caption={Task 1 Arduino code}]
// Motor A connections
int enA = 9;
int in1 = 8;
int in2 = 7;

// Motor B connections
int enB = 3;
int in3 = 5;
int in4 = 4;

void setup() {
	// Set all motor control pins to outputs
	pinMode(enA, OUTPUT);
	pinMode(enB, OUTPUT);
	pinMode(in1, OUTPUT);
	pinMode(in2, OUTPUT);
	pinMode(in3, OUTPUT);
	pinMode(in4, OUTPUT);
	
	// Turn off motors initially
	digitalWrite(in1, LOW);
	digitalWrite(in2, LOW);
	digitalWrite(in3, LOW);
	digitalWrite(in4, LOW);
}

void loop() {
	// Run Motor A for 5 seconds
	analogWrite(enA, 255);
	digitalWrite(in1, HIGH);
	digitalWrite(in2, LOW);
	delay(5000);
	digitalWrite(in1, LOW);
	digitalWrite(in2, LOW);
	
	// Run Motor B for 5 seconds
	analogWrite(enB, 255);
	digitalWrite(in3, HIGH);
	digitalWrite(in4, LOW);
	delay(5000);
	digitalWrite(in3, LOW);
	digitalWrite(in4, LOW);
	delay(1000);
	
	\end{lstlisting}
	
	\begin{lstlisting}[style=courier10, caption={Task 1 Arduino code}]
	// Run both motors for 10 seconds
	analogWrite(enA, 255);
	analogWrite(enB, 255);
	digitalWrite(in1, HIGH);
	digitalWrite(in2, LOW);
	digitalWrite(in3, HIGH);
	digitalWrite(in4, LOW);
	delay(10000);
	
	// Stop all motors
	digitalWrite(in1, LOW);
	digitalWrite(in2, LOW);
	digitalWrite(in3, LOW);
	digitalWrite(in4, LOW);
	delay(1000);
}


\end{lstlisting}

	\textbf{	Task No.:} 02 \\
	\textbf{	Task Name:} Implementing Conditional Activation in Timed Dual Motor Control\\
	
	\textbf{Objective:}
	
	
	\begin{itemize}
		\item To design a control system where Motor 2 activates only upon receiving a user input or
		signal after Motor 1 stops, followed by simultaneous motor operation for a fixed duration,
		in a continuous cycle.
		
	\end{itemize}
	
	
	
	\textbf{	Code:}
\begin{lstlisting}[style=courier10, caption={Task 2 Arduino code}]
	// Motor A connections
	int enA = 9;
	int in1 = 8;
	int in2 = 7;
	// Motor B connections
	int enB = 3;
	int in3 = 5;
	int in4 = 4;
	bool a = true;
	void setup() {
		// Set all the motor control pins to outputs
		pinMode(enA, OUTPUT);
		pinMode(enB, OUTPUT);
		pinMode(in1, OUTPUT);
		pinMode(in2, OUTPUT);
		pinMode(in3, OUTPUT);
		pinMode(in4, OUTPUT);
		pinMode(12, INPUT);
		// Turn off motors - Initial state
		digitalWrite(in1, LOW);
		digitalWrite(in2, LOW);
		digitalWrite(in3, LOW);
		digitalWrite(in4, LOW);
	}
		\end{lstlisting}
	
	\begin{lstlisting}[style=courier10, caption={Task 2 Arduino code}]
	void loop() {
		if (a == true) {
			analogWrite(enA, 255);
			digitalWrite(in1, HIGH);
			digitalWrite(in2, LOW);
			delay(5000);
			a = false;
			digitalWrite(in1, LOW);
			digitalWrite(in2, LOW);
		}
		if (digitalRead(12) == HIGH) {
			analogWrite(enB, 255);
			// Motor 2 run 5 sec
			digitalWrite(in3, HIGH);
			digitalWrite(in4, LOW);
			delay(5000);
			digitalWrite(in3, LOW);
			digitalWrite(in4, LOW);
			delay(1000);
	
			// Now run both motors for 10 seconds
			analogWrite(enA, 255);
			analogWrite(enB, 255);
			digitalWrite(in1, HIGH);
			digitalWrite(in2, LOW);
			digitalWrite(in3, HIGH);
			digitalWrite(in4, LOW);
			delay(10000);
			
			digitalWrite(in1, LOW);
			digitalWrite(in2, LOW);
			digitalWrite(in3, LOW);
			digitalWrite(in4, LOW);
			delay(1000);
			a = true;
		}
	}
	
\end{lstlisting}

	
	\textbf{	Task No.:} 03 \\
	\textbf{	Task Name:} Developing an Interlock-Based Safety Control in Dual Motor Sequencing\\
	
	\textbf{Objective:}
	
	
	\begin{itemize}
		\item To integrate a safety mechanism into the motor control sequence where the entire process
		halts upon detecting a danger signal, while preserving conditional and timed motor
		operation.
	\end{itemize}
	
	
	
	
	
\newpage
	
	\textbf{	Code:}
	\begin{lstlisting}[style=courier10, caption={Task 3 Arduino code}]
	// Motor A connections & Motor B connections
	int enA = 9;
	int in1 = 8;
	int in2 = 7;
	int enB = 3;
	int in3 = 5;
	int in4 = 4;
	const int dangerPin = 12; // Danger button pin
	bool stageA_done = false;
	bool paused = false;
	unsigned long lastTime = 0;
	int state = 0;
	
	void setup() {
		// Set motor pins as outputs
		pinMode(enA, OUTPUT);
		pinMode(enB, OUTPUT);
		pinMode(in1, OUTPUT);
		pinMode(in2, OUTPUT);
		pinMode(in3, OUTPUT);
		pinMode(in4, OUTPUT);
		pinMode(dangerPin, INPUT);
		stopAllMotors();
	}
	void loop() {
		// Check danger signal
		if (digitalRead(dangerPin) == HIGH) {
			stopAllMotors();
			paused = true;
			return;  // Exit loop early if paused
		}
		if (paused) {
			// Wait until dangerPin is LOW to resume
			if (digitalRead(dangerPin) == LOW) {
				paused = false;
				delay(200);  // Debounce
			} else {
				return;
			}
		}
		switch (state) {
			case 0:  // Run Motor A for 5 sec (only once per full cycle)
			analogWrite(enA, 255);
			digitalWrite(in1, HIGH);
			digitalWrite(in2, LOW);
			lastTime = millis();
			state = 1;
			break;
				
			case 1:  // Wait while Motor A runs
			if (millis() - lastTime >= 5000) {
				digitalWrite(in1, LOW);
				digitalWrite(in2, LOW);
				state = 2;
			}
			break;
		\end{lstlisting}
		
		\begin{lstlisting}[style=courier10, caption={Task 3 Arduino code}]
			case 2:  // Wait for danger button press to begin next phase
			if (digitalRead(dangerPin) == HIGH) {
				paused = true;
				return;
			}
			
			// Now run Motor B for 5 sec
			analogWrite(enB, 255);
			digitalWrite(in3, HIGH);
			digitalWrite(in4, LOW);
			lastTime = millis();
			state = 3;
			break;
			
			case 3:  // Wait while Motor B runs
			if (millis() - lastTime >= 5000) {
				digitalWrite(in3, LOW);
				digitalWrite(in4, LOW);
				delay(1000);
				state = 4;
			}
			break;
			
			case 4:  // Run both motors for 10 sec
			analogWrite(enA, 255);
			analogWrite(enB, 255);
			digitalWrite(in1, HIGH);
			digitalWrite(in2, LOW);
			digitalWrite(in3, HIGH);
			digitalWrite(in4, LOW);
			lastTime = millis();
			state = 5;
			break;
			
			case 5:  // Wait while both motors run
			if (millis() - lastTime >= 10000) {
				stopAllMotors();
				delay(1000);
				state = 0;  // Reset for next cycle
			}
			break;
		}
	}
	void stopAllMotors() {
		digitalWrite(in1, LOW);
		digitalWrite(in2, LOW);
		digitalWrite(in3, LOW);
		digitalWrite(in4, LOW);
	}
	
	\end{lstlisting}
	

\section*{Results:}
\textbf{Task 1 :} \begin{enumerate}
	\item Motor A: 5 seconds
	
	\item 	Motor B: 5 seconds
	
	\item 	Both Motors: 10 seconds

\end{enumerate}
\textbf{Task 2 :}
\begin{enumerate}
	\item After Motor A finishes:
	
\item	Wait for input to activate Motor B.
	
\item	Then proceed to dual motor run for 10 seconds.
\end{enumerate}
\textbf{Task 3 :}
\begin{enumerate}
	\item State 0: Run Motor A (5s)
	
\item	State 1: Wait for completion
	
\item	State 2: Wait for signal to run Motor B
	
\item	State 3: Run Motor B (5s)
	
\item	State 4: Run both motors (10s)
	
\item	State 5: Complete cycle and reset
\end{enumerate}

	\section*{Discussions:}
	In the shop practice , from task 1 it was clearly seen that the system was fully automatic as there was no user input to perform specific operation.\\
From the task 2 it was seen that one of the motor run for 5 seconds and wait for user input button to be pressed. Here we introduced a Boolean flag 'a' to control first-run condition.
After pressing the button the conditional and timed operations were performed and this process was continued by changing the flag states.\\
From task 3 it was seen that due to additional safety pin, here we introduced millis() instead of delay() for non-blocking timing operations. As there was two user input to perform, we used switch conditions where five states were used to check external input and perform the conditional operations.

	\section*{Conclusion:}
	Task 1 was fully automatic, timed sequence operations and there was no user input to perform. It was simple and cyclic.
	Task 2 was conditional and timed operations which was interactive due to user input.But due to delay() functions it didn't detect user input during delay operation. 
	Task 3 was safety interlock operation which stops on danger signal. It was responsive. Due to millis() functions, it was non-blocking operation. To preserve the previous operations , some flag was introduced. It was safe and complex logical operations.
	\section*{Reference:}
	1. https://lastminuteengineers.com/l293d-dc-motor-arduino-tutorial/
	
	\newpage


\end{document}

