\documentclass[a4paper,12pt]{article}

\usepackage{graphicx} % Required for inserting images
\usepackage{amsmath,amssymb,amsfonts}
\usepackage{subcaption}
% -----------------------
% Package Imports
% -----------------------

% Set page margins
\usepackage[a4paper, top=1in, bottom=0.8in, left=1.1in, right=0.8in]{geometry}

% Use Times New Roman font
\usepackage{times}

% Add page numbering
\pagestyle{plain}
\usepackage{multirow}
% Enable graphics inclusion
\usepackage{graphicx}
\usepackage{float}
% Enable code listings
\usepackage{listings}
\usepackage{xcolor} % For customizing code colors

% Define MATLAB style for listings
\lstdefinestyle{vscode-light}{
	language=Matlab,
	basicstyle=\ttfamily\footnotesize,
	keywordstyle=\color{blue},
	commentstyle=\color{gray},
	stringstyle=\color{red},
	numberstyle=\tiny\color{black},
	numbersep=5pt,
	frame=single,
	backgroundcolor=\color{white!10},
	breaklines=true,
	captionpos=b,
	tabsize=4,
	showstringspaces=false,
	numbers=left,  % Enable line numbering on the left
	stepnumber=1,  % Line numbers increment by 1
	numberfirstline=true, % Number the first line
}
\setlength{\parindent}{0pt}
\begin{document}
	

		
		\section*{Quiz Solutions}
		
		\begin{enumerate}
			\item Bend the leads of components at \textbf{45} degree angle with PCB.
			\begin{enumerate}
				\item 40
				\item 50
				\item 35
				\item \textbf{45}
			\end{enumerate}
			
			\item What measurements can you make using a oscilloscope?
			\begin{enumerate}
				\item Current, Voltage and resistance
				\item \textbf{Voltage, period and frequency}
				\item Time, frequency and resistance
				\item Power, distance and frequency
			\end{enumerate}
			
			\item What will be the power dissipation across a silicon diode carrying a current of 50 mA.
			\begin{enumerate}
				\item 25 mW
				\item 50 mW
				\item \textbf{35 mW} (Assuming a typical forward voltage drop of 0.7V for a silicon diode, Power = Voltage $\times$ Current = 0.7V $\times$ 50mA = 35mW)
				\item 100 mW
			\end{enumerate}
			
			\item Which among the below mentioned approaches belongs to the category of In-circuit Testing?
			\begin{enumerate}
				\item Impedance Testing
				\item Component Testing
				\item Apply Signal and check output
				\item \textbf{All of the above}
			\end{enumerate}
			
			\item What language is a typical Arduino code based on?
			\begin{enumerate}
				\item Assembly Code
				\item Python
				\item Java
				\item \textbf{C/C++}
			\end{enumerate}
			
			\item Arduino Codes are referred to as \textbf{sketches} in the Arduino IDE.
			\begin{enumerate}
				\item \textbf{sketches}
				\item drawings
				\item links
				\item notes
			\end{enumerate}
			
			\item What is the use of the Vin pin present on some Arduino Boards?
			\begin{enumerate}
				\item To ground the Arduino Board
				\item \textbf{To power the Arduino Board}
				\item To provide a 5V output
				\item Is used for plugging in 3V supply
			\end{enumerate}
			
			\item What will be the output of the following Arduino code?
			\begin{verbatim}
				#define X 10;
				void setup(){
					X=0;
					Serial.begin(9600);
					Serial.print(X);
				}
				void loop(){
					//Do nothing...
				}
			\end{verbatim}
			\begin{enumerate}
				\item 0xAB
				\item 0xa
				\item 0
				\item \textbf{Error} (The $`\# define X 10;`$ creates a preprocessor macro. The subsequent `X=0;` attempts to assign a value to a macro, which is not allowed. This would result in a compilation error.)
			\end{enumerate}
	
	
		
		\begin{enumerate}
			\item[9.] What describes the circuit connections in the diagram of PCB design?
			\begin{enumerate}
				\item Schematic Capture
				\item PCB Layout
				\item Equipment's
				\item a \& b
			\end{enumerate}
			\textbf{Answer: d) a \& b}
		\end{enumerate}
		
			\item	\textbf{Question:} What is the output of "pin1" if "pin2" is sent "1011" where 1 is 5V and 0 is 0V?
		\begin{verbatim}
			int pin1 = 12;
			int pin2 = 11;
			void setup(){
				pinMode(pin1, OUTPUT);
				pinMode(pin2, INPUT);
			}
			void loop() {
				if(digitalRead(pin2)==1) {
					digitalWrite(pin1,LOW);
				}
				else if(digitalRead(pin2)==0) {
					digitalWrite(pin1,HIGH);
				}
			}
		\end{verbatim}
		The options are:
		a) 1110
		b) 0100
		c) 1111
		d) 1011
		
		\textbf{Solution:}
		The Arduino code sets pin1 as an output and pin2 as an input.
		The loop() function implements a simple inverter logic:
		\begin{itemize}
			\item If pin2 reads HIGH (1 or 5V), pin1 is set to LOW (0V).
			\item If pin2 reads LOW (0 or 0V), pin1 is set to HIGH (5V).
		\end{itemize}
		This means the output on pin1 will always be the inverse of the input on pin2.
		
		Given that pin2 is sent "1011":
		\begin{itemize}
			\item For the first bit '1' (HIGH), pin1 outputs LOW ('0').
			\item For the second bit '0' (LOW), pin1 outputs HIGH ('1').
			\item For the third bit '1' (HIGH), pin1 outputs LOW ('0').
			\item For the fourth bit '1' (HIGH), pin1 outputs LOW ('0').
		\end{itemize}
		Therefore, the output on pin1 will be "0100".
		
		\textbf{Answer: b) 0100}
		
		\item What is the output of the program given below if a voltage of 5V is supplied to the pin corresponding to the A0 pin on an Arduino UNO?
		\begin{verbatim}
			void setup(){
				Serial.begin(9600);
				pinMode(A0, INPUT);
			}
			void loop(){
				int s = analogRead(A0);
				Serial.println(s);
			}
		\end{verbatim}
		\begin{enumerate}
			\item 0
			\item 1024
			\item null
			\item Error
		\end{enumerate}
		\textbf{Answer: b) 1024 (An Arduino's analogRead() function returns a 10-bit integer value (0-1023) representing the voltage. With a 5V supply and 5V input, it would read the maximum value, which is 1023 (or often approximated to 1024 as the number of divisions).)}
		
		
		
		
		\item What type of signal does the analogWrite() function output?
		\begin{enumerate}
			\item Pulse Code Modulated Signal
			\item Frequency Modulated Signal
			\item \textbf{Pulse Width Modulated Signal}
			\item Pulse Amplitude Modulated Signal
		\end{enumerate}
		
		\item A \textbf{LED} Arduino UNO is connected to pin 13 on an Analog-to-Digital Convertor (ADC)
		\begin{enumerate}
			\item Buffer
			\item \textbf{LED}
			\item Digital-to-Analog Convertor (DAC)
		\end{enumerate}
		(Note: Pin 13 on an Arduino UNO typically has an onboard LED connected to it. The phrasing "connected to pin 13 on an Analog-to-Digital Convertor (ADC)" is slightly unusual as Pin 13 is a digital pin, not an analog input to an ADC, but an LED is the most fitting answer for what is usually on pin 13.)
		
		\item While you upload a sketch to the Arduino UNO, the Tx and RX LEDs should... This question is required.
		\begin{enumerate}
			\item Be constantly ON
			\item \textbf{Blink Rapidly}
			\item Be OFF
		\end{enumerate}
		
		\item Traces and planes utilized in PCB designing comprises of \textbf{Copper}?
		\begin{enumerate}
			\item Lead
			\item \textbf{Copper}
			\item Silver
			\item Titanium
		\end{enumerate}
		
		\item How many conducting layers are present in Single-sided PCB?
		\begin{enumerate}
			\item \textbf{One}
			\item Two
			\item Three
			\item Four
		\end{enumerate}
		
		\item Conducting layer added on bottom and top of \textbf{Double-sided} PCBs.
		\begin{enumerate}
			\item Single-sided
			\item \textbf{Double-sided}
			\item Multilayer
			\item Rigid
		\end{enumerate}
		
		\item What is calculated based on various circuit's current necessities?
		\begin{enumerate}
			\item Impedances
			\item Number of Layers
			\item \textbf{Trace Width}
			\item Plane Width
		\end{enumerate}
		
		\item Color code for 1 ohm resistance is -
		\begin{enumerate}
			\item Black, Brown, Gold
			\item \textbf{Brown, Black Gold}
			\item Both of them
			\item None of them
		\end{enumerate}
		
		\item What describes the circuit connections in the diagram of PCB design?
		\begin{enumerate}
			\item Schematic Capture
			\item PCB Layout
			\item Equipment's
			\item \textbf{a \& b}
		\end{enumerate}
		
		\item What connects pins of components in PCB?
		\begin{enumerate}
			\item Traces
			\item Planes
			\item \textbf{Nets}
			\item Points
		\end{enumerate}
			\end{enumerate}
		\end{document}
		
		

		
	
		
	
		
	\end{document}