\documentclass[a4paper,12pt]{article}

% -----------------------
% Package Imports
% -----------------------

% Page margins
\usepackage[a4paper, top=1in, bottom=0.8in, left=1.1in, right=0.8in]{geometry}

% Fonts
\usepackage{times}  % Times New Roman

% Math symbols
\usepackage{amsmath,amssymb,amsfonts}

% For figures and subfigures
\usepackage{graphicx}
\usepackage{subcaption}

% For listing code
\usepackage{listings}
\usepackage{xcolor}

% For table formatting
\usepackage{multirow}

% To place figures and code listings precisely
\usepackage{float}

% Line numbers and code formatting style
\lstdefinestyle{vscode-light}{
	language=Matlab,
	basicstyle=\ttfamily\footnotesize,
	keywordstyle=\color{blue},
	commentstyle=\color{gray},
	stringstyle=\color{red},
	numberstyle=\tiny\color{black},
	numbersep=5pt,
	frame=single,
	backgroundcolor=\color{white!10},
	breaklines=true,
	captionpos=b,
	tabsize=4,
	showstringspaces=false,
	numbers=left,
	stepnumber=1,
	numberfirstline=true,
}

% Remove paragraph indentation
\setlength{\parindent}{0pt}

% Page style
\pagestyle{plain}



\begin{document}
	
	\section*{Quiz Solutions}
	
	\begin{enumerate}
		\item Bend the leads of components at \textbf{45} degree angle with PCB.
		\begin{enumerate}
			\item 40
			\item 50
			\item 35
			\item \textbf{45}
		\end{enumerate}
		\textbf{Explanation:} Components leads are typically bent at 45° for proper PCB insertion and soldering.
		
		\item What measurements can you make using a oscilloscope?
		\begin{enumerate}
			\item Current, Voltage and resistance
			\item \textbf{Voltage, period and frequency}
			\item Time, frequency and resistance
			\item Power, distance and frequency
		\end{enumerate}
		\textbf{Explanation:} Oscilloscopes primarily measure voltage over time, from which period and frequency can be derived.
		
		\item What will be the power dissipation across a silicon diode carrying a current of 50 mA.
		\begin{enumerate}
			\item 25 mW
			\item 50 mW
			\item \textbf{35 mW}
			\item 100 mW
		\end{enumerate}
		\textbf{Explanation:} Power = Voltage × Current = 0.7V × 50mA = 35mW (using typical 0.7V drop for silicon diodes).
		
		\item Which among the below mentioned approaches belongs to the category of In-circuit Testing?
		\begin{enumerate}
			\item Impedance Testing
			\item Component Testing
			\item Apply Signal and check output
			\item \textbf{All of the above}
		\end{enumerate}
		\textbf{Explanation:} In-circuit testing can include impedance checks, component verification, and signal testing.
		
		\item What language is a typical Arduino code based on?
		\begin{enumerate}
			\item Assembly Code
			\item Python
			\item Java
			\item \textbf{C/C++}
		\end{enumerate}
		\textbf{Explanation:} Arduino programming language is based on C/C++ with simplified libraries.
		
		\item Arduino Codes are referred to as \textbf{sketches} in the Arduino IDE.
		\begin{enumerate}
			\item \textbf{sketches}
			\item drawings
			\item links
			\item notes
		\end{enumerate}
		\textbf{Explanation:} The Arduino IDE uses the term "sketches" for program files by convention.
		
		\item What is the use of the Vin pin present on some Arduino Boards?
		\begin{enumerate}
			\item To ground the Arduino Board
			\item \textbf{To power the Arduino Board}
			\item To provide a 5V output
			\item Is used for plugging in 3V supply
		\end{enumerate}
		\textbf{Explanation:} Vin is used to power the board with an external voltage (7-12V recommended).
		
		\item What will be the output of the following Arduino code?
		\begin{verbatim}
			#define X 10;
			void setup(){
				X=0;
				Serial.begin(9600);
				Serial.print(X);
			}
			void loop(){
				//Do nothing...
			}
		\end{verbatim}
		\begin{enumerate}
			\item 0xAB
			\item 0xa
			\item 0
			\item \textbf{Error}
		\end{enumerate}
		\textbf{Explanation:} The code will produce a compilation error because you cannot assign a value to a macro (X is defined as constant 10).
		
		\item What describes the circuit connections in the diagram of PCB design?
		\begin{enumerate}
			\item Schematic Capture
			\item PCB Layout
			\item Equipment's
			\item \textbf{a \& b}
		\end{enumerate}
		\textbf{Explanation:} Both schematic capture (logical connections) and PCB layout (physical implementation) describe circuit connections.
		
		\item What is the output of "pin1" if "pin2" is sent "1011" where 1 is 5V and 0 is 0V?
		\begin{verbatim}
			int pin1 = 12;
			int pin2 = 11;
			void setup(){
				pinMode(pin1, OUTPUT);
				pinMode(pin2, INPUT);
			}
			void loop() {
				if(digitalRead(pin2)==1) {
					digitalWrite(pin1,LOW);
				}
				else if(digitalRead(pin2)==0) {
					digitalWrite(pin1,HIGH);
				}
			}
		\end{verbatim}
		\begin{enumerate}
			\item 1110
			\item \textbf{0100}
			\item 1111
			\item 1011
		\end{enumerate}
		\textbf{Explanation:} The code implements an inverter - pin1 outputs the opposite of pin2's input (1011 becomes 0100).
		
		\item What is the output of the program given below if a voltage of 5V is supplied to the pin corresponding to the A0 pin on an Arduino UNO?
		\begin{verbatim}
			void setup(){
				Serial.begin(9600);
				pinMode(A0, INPUT);
			}
			void loop(){
				int s = analogRead(A0);
				Serial.println(s);
			}
		\end{verbatim}
		\begin{enumerate}
			\item 0
			\item \textbf{1023}
			\item null
			\item Error
		\end{enumerate}
		\textbf{Explanation:} The 10-bit ADC will return 1023 (maximum value) for 5V input (Arduino's reference voltage).
		
		\item What type of signal does the analogWrite() function output?
		\begin{enumerate}
			\item Pulse Code Modulated Signal
			\item Frequency Modulated Signal
			\item \textbf{Pulse Width Modulated Signal}
			\item Pulse Amplitude Modulated Signal
		\end{enumerate}
		\textbf{Explanation:} analogWrite() generates PWM (Pulse Width Modulation) signals on supported pins.
		
		\item A \textbf{LED} Arduino UNO is connected to pin 13 on an Analog-to-Digital Convertor (ADC)
		\begin{enumerate}
			\item Buffer
			\item \textbf{LED}
			\item Digital-to-Analog Convertor (DAC)
		\end{enumerate}
		\textbf{Explanation:} Pin 13 has a built-in LED on most Arduino boards (though it's a digital pin, not ADC input).
		
		\item While you upload a sketch to the Arduino UNO, the Tx and RX LEDs should...
		\begin{enumerate}
			\item Be constantly ON
			\item \textbf{Blink Rapidly}
			\item Be OFF
		\end{enumerate}
		\textbf{Explanation:} These LEDs blink during serial communication (uploading is serial communication with the bootloader).
		
		\item Traces and planes utilized in PCB designing comprises of \textbf{Copper}?
		\begin{enumerate}
			\item Lead
			\item \textbf{Copper}
			\item Silver
			\item Titanium
		\end{enumerate}
		\textbf{Explanation:} PCBs use copper for conductive layers due to its excellent conductivity and cost-effectiveness.
		
		\item How many conducting layers are present in Single-sided PCB?
		\begin{enumerate}
			\item \textbf{One}
			\item Two
			\item Three
			\item Four
		\end{enumerate}
		\textbf{Explanation:} Single-sided PCBs have conductive traces on only one side of the substrate.
		
		\item Conducting layer added on bottom and top of \textbf{Double-sided} PCBs.
		\begin{enumerate}
			\item Single-sided
			\item \textbf{Double-sided}
			\item Multilayer
			\item Rigid
		\end{enumerate}
		\textbf{Explanation:} Double-sided PCBs have copper layers on both sides of the substrate.
		
		\item What is calculated based on various circuit's current necessities?
		\begin{enumerate}
			\item Impedances
			\item Number of Layers
			\item \textbf{Trace Width}
			\item Plane Width
		\end{enumerate}
		\textbf{Explanation:} Trace width is determined by current requirements to prevent overheating and ensure proper current capacity.
		
		\item Color code for 1 ohm resistance is -
		\begin{enumerate}
			\item Black, Brown, Gold
			\item \textbf{Brown, Black, Gold}
			\item Both of them
			\item None of them
		\end{enumerate}
		\textbf{Explanation:} 1 ohm resistor is coded as Brown (1), Black (0), Gold (×0.1 multiplier).
		
		\item What connects pins of components in PCB?
		\begin{enumerate}
			\item Traces
			\item Planes
			\item \textbf{Nets}
			\item Points
		\end{enumerate}
		\textbf{Explanation:} Nets represent electrical connections between component pins in PCB design.
	\end{enumerate}
	
\end{document}
