\documentclass{article}
\usepackage{amsmath}
\usepackage{amsfonts}
\usepackage{amssymb}
\usepackage{graphicx}
\usepackage{listings}
\usepackage{xcolor}
\usepackage{geometry}

\geometry{a4paper, margin=1in}

\definecolor{codegreen}{rgb}{0,0.6,0}
\definecolor{codegray}{rgb}{0.5,0.5,0.5}
\definecolor{codepurple}{rgb}{0.58,0,0.82}
\definecolor{backcolour}{rgb}{0.98,0.98,0.98}

\lstdefinestyle{mystyle}{
	backgroundcolor=\color{backcolour},
	commentstyle=\color{codegreen},
	keywordstyle=\color{magenta},
	numberstyle=\tiny\color{codegray},
	stringstyle=\color{codepurple},
	basicstyle=\ttfamily\footnotesize,
	breakatwhitespace=false,
	breaklines=true,
	captionpos=b,
	keepspaces=true,
	numbers=left,
	numbersep=5pt,
	showspaces=false,
	showstringspaces=false,
	showtabs=false,
	tabsize=2
}

\lstset{style=mystyle}

\begin{document}
	
	\section*{Time Domain Analysis (MATLAB)}
	
	\subsection*{What is Time Domain Analysis?}
	Time domain analysis is used to study how a system responds to inputs over time. In control systems, it reveals how outputs like position or speed change due to step, ramp, or impulse inputs.
	
	\subsection*{Important Formulas}
	\begin{itemize}
		\item Natural frequency: $\omega_n = \sqrt{a}$
		\item Damping ratio: $\zeta = \frac{b}{2\omega_n}$
		\item Peak Time: $T_p = \frac{\pi}{\omega_n\sqrt{1 - \zeta^2}}$
		\item Rise Time (approx for underdamped): $T_r \approx \frac{1.8}{\omega_n}$
		\item Settling Time: $T_s = \frac{4}{\zeta\omega_n}$
		\item Max Overshoot: $M_p = e^{\frac{-\pi\zeta}{\sqrt{1 - \zeta^2}}} \times 100\%$
	\end{itemize}
	
	\section*{Problem 1}
	Given:
	\[
	\frac{C(s)}{R(s)} = \frac{9}{s^2 + 5s + 9}
	\]
	
	\begin{lstlisting}[language=Matlab, caption=Time response metrics without built-in functions]
		a = 9; b = 5; c = 9;
		wn = sqrt(c);
		z = b / (2*wn);
		tp = pi / (wn*sqrt(1-z^2));
		tr = 1.8 / wn;
		ts = 4 / (z*wn);
		mp = exp((-pi*z)/sqrt(1-z^2)) * 100;
		fprintf('Rise Time: %.3f\nPeak Time: %.3f\nSettling Time: %.3f\nOvershoot: %.2f%%\n', tr, tp, ts, mp);
	\end{lstlisting}
	
	\section*{Problem 2}
	Step response for various damping ratio $\zeta$:
	\[
	\frac{C(s)}{R(s)} = \frac{\omega_n^2}{s^2 + 2\zeta\omega_n s + \omega_n^2}, \quad \omega_n = 1.5
	\]
	\begin{lstlisting}[language=Matlab, caption=Step response vs damping ratio (manual)]
		% Natural frequency (assumed constant)
		wn = 5;
		
		% Time vector
		t = 0:0.005:5;
		
		% Prepare figure
		figure;
		sgtitle('Step Responses for Different \zeta Values');
		
		% Loop for 6 different zeta inputs
		for i = 1:6
		% Take zeta input from user
		zeta = input(['Enter value of zeta for case ' num2str(i) ': ']);
		
		% Define transfer function: H(s) = wn^2 / (s^2 + 2*zeta*wn*s + wn^2)
		num = [0 0 wn^2];
		den = [1 2*zeta*wn wn^2];
		
		% Calculate step response
		[y, ~] = step(tf(num, den), t);
		
		% Plot subplot
		subplot(3, 2, i);
		plot(t, y, 'b', 'LineWidth', 1.5);
		grid on;
		title(['\zeta = ' num2str(zeta)]);
		xlabel('Time (s)');
		ylabel('Response');
		end
		
	\end{lstlisting}
	\subsubsection{Direct Approch}
	\begin{lstlisting}[language=Matlab, caption=Step response vs damping ratio (manual)]
		t = 0:0.01:10;
		wn = 1.5;
		zeta = [-1, -0.5, 0, 0.2, 0.5, 1, 2];
		for i = 1:length(zeta)
		z = zeta(i);
		wd = wn * sqrt(abs(1 - z^2));
		if z < 1
		y = 1 - (1/sqrt(1-z^2))*exp(-z*wn*t).*sin(wd*t + acos(z));
		else
		y = 1 - exp(-wn*t); % approximation for overdamped
		end
		plot(t, y); hold on;
		end
		legend('z=-1','-0.5','0','0.2','0.5','1','2');
		title('Step response vs \zeta');
		xlabel('Time'); ylabel('Amplitude');
		grid on;
	\end{lstlisting}
	
	\section*{Problem 3}
	Given system:
	\[
	\frac{C(s)}{R(s)} = \frac{3}{s^2 + 3s + 3}
	\]
	
	\subsection*{a) Manual Step Response (using inverse Laplace)}
	\begin{lstlisting}[language=Matlab, caption=Manual step response using convolution]
		t = 0:0.01:10;
		imp = [1 zeros(1, length(t)-1)];
		dt = t(2) - t(1);
		a = 3; b = 3; c = 3;
		
		h = exp(-b/2*t).*sin(sqrt(c - (b^2)/4)*t);
		y_step = dt * cumsum(h);
		plot(t, y_step); title('Unit Step Response'); xlabel('t'); ylabel('y(t)');
	\end{lstlisting}
	
	\subsection*{b) State-Space System}
	\[
	\begin{bmatrix}
		\dot{x}_1 \\ \dot{x}_2
	\end{bmatrix}
	=
	\begin{bmatrix}
		-1 & -0.5\\
		1 & 0
	\end{bmatrix}
	\begin{bmatrix}
		x_1 \\ x_2
	\end{bmatrix}
	+
	\begin{bmatrix}
		0.5\\
		0
	\end{bmatrix} u,
	\quad
	y = [1 -1] \begin{bmatrix}
		x_1 \\ x_2
	\end{bmatrix}
	\]
	
	\begin{lstlisting}[language=Matlab, caption=Manual response of state-space]
		t = 0:0.01:10;
		A = [-1 -0.5; 1 0];
		B = [0.5; 0];
		C = [1 -1];
		x = [0; 0];
		dt = t(2) - t(1);
		y = zeros(size(t));
		for i = 1:length(t)
		dx = A*x + B*1;
		x = x + dx*dt;
		y(i) = C*x;
		end
		plot(t, y); title('State-Space Unit Step'); xlabel('t'); ylabel('y');
	\end{lstlisting}
	
\end{document}
