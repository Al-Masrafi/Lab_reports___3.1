\documentclass[a4paper,12pt]{article}

\usepackage{graphicx} % Required for inserting images
\usepackage{amsmath,amssymb,amsfonts}
\usepackage{subcaption}
% -----------------------
% Package Imports
% -----------------------

% Set page margins
\usepackage[a4paper, top=1in, bottom=0.8in, left=1.1in, right=0.8in]{geometry}

% Use Times New Roman font
\usepackage{times}
\usepackage{pdfpages} % Include PDF files





% Set page margins
\usepackage[a4paper, top=1in, bottom=0.8in, left=1.1in, right=0.8in]{geometry}


% Add page numbering
\pagestyle{plain}

% Enable graphics inclusion
\usepackage{graphicx}
\usepackage{float}
% Enable code listings
\usepackage{listings}
\usepackage{xcolor} % For customizing code colors
\setlength{\parindent}{0pt}
\usepackage{multirow}

\setlength{\parindent}{0pt}
\usepackage{titlesec} % To customize section font size
\titleformat{\section}
{\normalfont\fontsize{14}{16}\bfseries}{\thesection}{1em}{}

\titleformat{\subsection}
{\normalfont\fontsize{14}{16}\bfseries}{\thesubsection}{1em}{}
\title{Induction Motor Equivalent Circuit and Performance Analysis}
\author{}
\date{}

\begin{document}
	
	\maketitle
	

\section*{6-1. What are slip and slip speed in an induction motor?}
\textbf{Slip} ($s$) is the difference between synchronous speed ($N_s$) and rotor speed ($N_r$), expressed as a fraction of synchronous speed:
\begin{equation}
	s = \frac{N_s - N_r}{N_s}
\end{equation}
\textbf{Slip speed} is the actual speed difference:
\begin{equation}
	N_{\text{slip}} = N_s - N_r
\end{equation}

\section*{6-2. How does an induction motor develop torque?}
An induction motor develops torque due to the interaction of the rotating magnetic field from the stator and the currents induced in the rotor conductors. The induced current produces its own magnetic field, and the interaction results in a torque as per Lorentz force law:
\begin{equation}
	\vec{F} = I(\vec{L} \times \vec{B})
\end{equation}

\section*{6-3. Why is it impossible for an induction motor to operate at synchronous speed?}
At synchronous speed ($s = 0$), there is no relative motion between rotor and stator field, hence no induced current and thus no torque. Therefore, induction motors cannot run at synchronous speed.

\section*{6-4. Sketch and explain the shape of a typical induction motor torque-speed characteristic curve.}
\begin{figure}[H]
	\centering
	%		\includegraphics[width=0.6\textwidth]{torque_speed_curve.png}
	\caption{Typical Torque-Speed Characteristic of an Induction Motor}
\end{figure}
The curve starts at zero torque (standstill), rises to a peak (pull-out torque), and drops as speed approaches synchronous speed.

\section*{6-5. What equivalent circuit element has the most direct control over the speed at which the pullout torque occurs?}
The rotor resistance $R_2$ in the equivalent circuit directly affects the slip at which pull-out torque occurs. Higher $R_2$ shifts the peak torque to higher slip.

\section*{6-6. What is a deep-bar cage rotor? Why is it used? What NEMA design class(es) can be built with it?}
A deep-bar rotor uses deep rotor bars to create a skin effect, increasing resistance at starting and reducing it at running condition. It is used to improve starting torque and is suitable for NEMA B and C designs.

\section*{6-7. What is a double-cage cage rotor? Why is it used? What NEMA design class(es) can be built with it?}
Consists of two layers: outer cage (high resistance) for starting torque, and inner cage (low resistance) for efficiency during running. Used in NEMA B and C designs.

\section*{6-8. Describe the characteristics and uses of wound-rotor induction motors and of each NEMA design class of cage motors.}
\textbf{Wound-rotor motors:} External resistors can be added for speed control and high starting torque.

\textbf{NEMA Classes:}
\begin{itemize}
	\item \textbf{A:} Normal starting torque, high starting current
	\item \textbf{B:} Normal torque, lower starting current
	\item \textbf{C:} High starting torque, low starting current
	\item \textbf{D:} Very high starting torque, high slip
\end{itemize}

\section*{6-9. Why is the efficiency of an induction motor (wound-rotor or cage) so poor at high slips?}
At high slips, rotor current increases, causing high copper losses ($I^2R$ losses), reducing efficiency.

\section*{6-10. Name and describe four means of controlling the speed of induction motors.}
\begin{enumerate}
	\item Frequency control (V/f)
	\item Pole changing
	\item Rotor resistance (wound-rotor)
	\item Variable voltage
\end{enumerate}

\section*{6-11. Why is it necessary to reduce the voltage applied to an induction motor as electrical frequency is reduced?}
To maintain constant flux ($\phi \propto \frac{V}{f}$), voltage must decrease with frequency, else core saturation occurs.

\section*{6-12. Why is terminal voltage speed control limited in operating range?}
Limited range due to torque reduction at low voltages and overheating risks.

\section*{6-13. What are starting code factors? What do they say about the starting current of an induction motor?}
Code letters indicate locked-rotor kVA per HP. Higher code letters imply higher starting currents.

\section*{6-14. How does a resistive starter circuit for an induction motor work?}
Inserts resistors in series with stator to reduce voltage during start, then bypasses them.

\section*{6-15. What information is learned in a locked-rotor test?}
Provides rotor resistance and leakage reactance data. Performed with rotor locked.

\section*{6-16. What information is learned in a no-load test?}
Determines magnetizing reactance and core losses by running motor without load.

\section*{6-17. What actions are taken to improve the efficiency of modern high-efficiency induction motors?}
\begin{itemize}
	\item Better lamination steel
	\item Optimized design
	\item Lower resistance windings
	\item Tighter tolerances
\end{itemize}

\section*{6-18. What controls the terminal voltage of an induction generator operating alone?}
Controlled by the connected capacitor bank or grid excitation.

\section*{6-19. For what applications are induction generators typically used?}
\begin{itemize}
	\item Wind turbines
	\item Mini hydro plants
	\item Regenerative braking systems
\end{itemize}

\section*{6-20. How can a wound-rotor induction motor be used as a frequency changer?}
By varying rotor resistance and connecting external circuits, it can change frequency and act as a frequency converter.

\section*{6-21. How do different voltage-frequency patterns affect the torque-speed characteristics of an induction motor?}
\textbf{V/f control} keeps torque consistent. Reducing voltage without reducing frequency decreases torque; increasing voltage alone may saturate the core.

\section*{6-22. Describe the major features of the solid-state induction motor drive featured in Section 6.10.}
Uses inverter + rectifier to convert AC to DC to variable-frequency AC. Enables efficient speed control.

\section*{6-23. Two 480-V, 100-hp induction motors are manufactured. One is designed for 50-Hz operation, and one is designed for 60-Hz operation, but they are otherwise similar. Which of these machines is larger?}
Lower frequency (50 Hz) motor must be larger to produce the same power due to lower flux speed. Thus, 50 Hz motor is physically larger.

\section*{6-24. An induction motor is running at the rated conditions. If the shaft load is now increased, how do the following quantities change?}
\begin{enumerate}
	\item[(a)] Mechanical speed: Decreases
	\item[(b)] Slip: Increases
	\item[(c)] Rotor induced voltage: Increases
	\item[(d)] Rotor current: Increases
	\item[(e)] Rotor frequency: Increases
	\item[(f)] Power delivered to rotor (P$_{\text{Re}}$): Increases
	\item[(g)] Synchronous speed: Unchanged (depends on supply frequency)
\end{enumerate}

\newpage
	\begin{figure}[H]
		\centering
	
		\label{fig:screenshot001}
		
		\includegraphics[width=1\textwidth]{"D:/DOWNLOAD 2024 V2/LATEX FILE/3.1/screenshot001"}
			
	\end{figure}
		\section*{Given Data:}
	\begin{itemize}
		\item Rated voltage: $208\ \text{V}$ (line-line, Y-connected)
		\item Frequency: $f = 60\ \text{Hz}$
		\item Rated power: $25\ \text{hp} = 18650\ \text{W}$
		\item Poles: $P = 6$
		\item No-load test: $V_{nl} = 208\ \text{V}$, $I_{nl} = 24.0\ \text{A}$, $P_{nl} = 1400\ \text{W}$
		\item Blocked-rotor test: $V_{br} = 24.6\ \text{V}$, $I_{br} = 64.5\ \text{A}$, $P_{br} = 2200\ \text{W}$, $f_{br} = 15\ \text{Hz}$
		\item DC test: $V_{dc} = 13.5\ \text{V}$, $I_{dc} = 64\ \text{A}$
	\end{itemize}
	
	\section*{(a) Equivalent Circuit Parameters}
	
	The per-phase voltage for a Y-connected system:
	\[
	V_{ph} = \frac{208}{\sqrt{3}} = 120.09\ \text{V}
	\]
	
	\subsubsection*{From DC Test: Stator Resistance \( R_1 \)}
	\[
	R_{1} = \frac{V_{dc}}{2I_{dc}} = \frac{13.5}{2 \times 64} = 0.1055\ \Omega
	\]
	
	\subsubsection*{From No-load Test: Magnetizing Branch \( R_c, X_m \)}
	Per-phase values:
	\[
	P_{nl,ph} = \frac{1400}{3} = 466.67\ \text{W},\quad I_{nl,ph} = \frac{24.0}{\sqrt{3}} = 13.86\ \text{A}
	\]
	\[
	I_w = \frac{P_{nl,ph}}{V_{ph}} = \frac{466.67}{120.09} = 3.885\ \text{A},\quad
	I_m = \sqrt{I_{nl,ph}^2 - I_w^2} = \sqrt{13.86^2 - 3.885^2} = 13.3\ \text{A}
	\]
	\[
	R_c = \frac{V_{ph}}{I_w} = \frac{120.09}{3.885} = 30.92\ \Omega,\quad
	X_m = \frac{V_{ph}}{I_m} = \frac{120.09}{13.3} = 9.03\ \Omega
	\]
	
	\subsubsection*{From Blocked Rotor Test: \( R_1 + R_2', X_1 + X_2' \)}
	Per-phase:
	\[
	V_{br,ph} = \frac{24.6}{\sqrt{3}} = 14.2\ \text{V},\quad I_{br,ph} = \frac{64.5}{\sqrt{3}} = 37.25\ \text{A}
	\]
	\[
	P_{br,ph} = \frac{2200}{3} = 733.3\ \text{W},\quad
	Z_{br} = \frac{V_{br,ph}}{I_{br,ph}} = \frac{14.2}{37.25} = 0.381\ \Omega
	\]
	\[
	R_{eq} = \frac{P_{br,ph}}{I_{br,ph}^2} = \frac{733.3}{(37.25)^2} = 0.528\ \Omega
	\]
	\[
	X_{eq} = \sqrt{Z_{br}^2 - R_{eq}^2} = \sqrt{0.381^2 - 0.528^2} = \text{Imaginary (inconsistent)}
	\]
	
	Since $R_1$ was obtained from DC test, we use:
	\[
	R_2' = R_{eq} - R_1 = 0.528 - 0.1055 = 0.4225\ \Omega
	\]
	Assume symmetrical leakage reactance:
	\[
	X_1 = X_2' = \frac{X_{eq}}{2}
	\]
	
	\section*{(b) Equivalent Circuit Diagram}
	

	\section*{(c) Rotational Loss}
	
	Rotational loss = No-load input power $-$ stator copper loss $-$ core loss
	
	\[
	P_{rot} = P_{nl} - 3I_{nl,ph}^2R_1 - \frac{V_{ph}^2}{R_c}
	\]
	\[
	= 1400 - 3(13.86)^2(0.1055) - \frac{(120.09)^2}{30.92} = 1400 - 60.5 - 466.7 = 872.8\ \text{W}
	\]
	
	\section*{(d) Slip at Pull-out Torque}
	
	Approximate value for Class B induction motor:
	\[
	s_{max} \approx \frac{R_2'}{\sqrt{R_1^2 + (X_1 + X_2')^2}} \approx \frac{0.4225}{\sqrt{(0.1055)^2 + (0.381)^2}} = \frac{0.4225}{0.395} \approx 1.07
	\]
	
	But physically maximum slip is $\leq 1$, hence $s_{max} \approx 1$
	
	\section*{(e) Pullout Torque \( T_{max} \)}
	
	Synchronous speed:
	\[
	n_s = \frac{120f}{P} = \frac{120 \times 60}{6} = 1200\ \text{rpm}
	\]
	
	\[
	\omega_s = \frac{2\pi n_s}{60} = \frac{2\pi \times 1200}{60} = 125.66\ \text{rad/s}
	\]
	
	\[
	T_{max} = \frac{3V_{ph}^2}{\omega_s \cdot 2[(R_1 + \sqrt{R_1^2 + (X_1 + X_2')^2})]} = \text{Complex expression; simplified}
	\]
	
	Use typical formula:
	\[
	T_{max} \approx \frac{3V_{ph}^2}{\omega_s(2X_{eq})} = \frac{3(120.09)^2}{125.66 \cdot (2 \cdot 0.381)} = 113.5\ \text{Nm}
	\]
	
	\section*{(f) Output Power at 5\% Slip}
	
	\subsubsection*{Air-gap Power \( P_{AG} \)}
	\[
	P_{AG} = \frac{P_{conv}}{1 - s}
	\]
	
	Let $P_{conv} = \text{rated output} = 18650\ \text{W}$, $s = 0.05$
	
	\[
	P_{AG} = \frac{18650}{1 - 0.05} = \frac{18650}{0.95} = 19631.6\ \text{W}
	\]
	
	\subsubsection*{Converted Power}
	
	\[
	P_{conv} = (1 - s)P_{AG} = 0.95 \cdot 19631.6 = 18650\ \text{W}
	\]
	
	\section*{Definitions}
	
	\begin{itemize}
		\item \textbf{Slip (s)}: The difference between synchronous and rotor speed, $s = \frac{n_s - n_r}{n_s}$
		\item \textbf{Air-gap Power \( P_{AG} \)}: Power transferred across the air gap to the rotor.
		\item \textbf{Converted Power \( P_{conv} \)}: The mechanical power developed in the rotor.
		\item \textbf{Pull-out Torque}: Maximum torque an induction motor can develop before stalling.
	\end{itemize}
	
	\newpage
	\section*{Theory of Equivalent Circuit of an Induction Motor}
	
	The equivalent circuit of an induction motor is an electrical representation of its behavior, analogous to a transformer with a rotating secondary. It allows us to analyze the performance of the motor using circuit analysis techniques.
	
	\section{Equivalent Circuit Overview}
	
	An induction motor’s per-phase equivalent circuit is divided into three main parts:
	\begin{enumerate}
		\item \textbf{Stator impedance:} Represents the stator resistance $R_1$ and leakage reactance $X_1$.
		\item \textbf{Magnetizing branch:} Consists of a core loss resistance $R_c$ (to model iron losses) and magnetizing reactance $X_m$ (to model the magnetizing current).
		\item \textbf{Rotor circuit referred to stator side:} Contains the rotor resistance $R_2'$ and leakage reactance $X_2'$, with the resistance divided by slip $s$ to model torque and power transfer.
	\end{enumerate}
	
	\section{Tests to Determine Parameters}
	
	To determine the parameters of the equivalent circuit, the following tests are performed:
	
	\subsection*{1. DC Test}
	Used to find the stator resistance $R_1$:
	\[
	R_1 = \frac{V_{dc}}{2I_{dc}}
	\]
	The factor of 2 assumes both stator windings (in series) are measured.
	
	\subsection*{2. No-load Test}
	This test is done at rated voltage and frequency with the rotor uncoupled (free to spin). It gives:
	\begin{itemize}
		\item Core loss resistance $R_c$
		\item Magnetizing reactance $X_m$
	\end{itemize}
	
	Using:
	\[
	I_{nl} = \text{No-load line current}, \quad V_{ph} = \text{Phase voltage}
	\]
	\[
	I_w = \frac{P_{nl}/3}{V_{ph}}, \quad I_m = \sqrt{I_{nl}^2 - I_w^2}
	\]
	\[
	R_c = \frac{V_{ph}}{I_w}, \quad X_m = \frac{V_{ph}}{I_m}
	\]
	
	\subsection*{3. Blocked Rotor Test}
	Conducted by locking the rotor and applying a reduced voltage to limit current. This test is similar to a short-circuit test in transformers and provides:
	\begin{itemize}
		\item Total leakage impedance: $Z_{br} = R_{eq} + jX_{eq}$
		\item Combined stator and rotor resistance and reactance: $R_1 + R_2'$, $X_1 + X_2'$
	\end{itemize}
	\[
	R_{eq} = \frac{P_{br}/3}{I_{br}^2}, \quad Z_{br} = \frac{V_{br}}{I_{br}}, \quad X_{eq} = \sqrt{Z_{br}^2 - R_{eq}^2}
	\]
	
	Assuming symmetry:
	\[
	R_2' = R_{eq} - R_1, \quad X_1 = X_2' = \frac{X_{eq}}{2}
	\]
	
	\section{Final Equivalent Circuit}
	
	The final per-phase equivalent circuit consists of:
	\begin{itemize}
		\item Stator resistance $R_1$ and reactance $X_1$
		\item Parallel magnetizing branch $R_c \parallel jX_m$
		\item Rotor resistance $R_2'/s$ and reactance $X_2'$
	\end{itemize}
	
	\begin{figure}[H]
		\centering
%		\includegraphics[width=0.7\textwidth]{https://i.imgur.com/f1DpH6j.png}
		\caption{Per-phase approximate equivalent circuit of an induction motor}
	\end{figure}
	
	\section{Importance of Equivalent Circuit}
	
	\begin{itemize}
		\item Helps to analyze performance parameters like efficiency, torque, power factor.
		\item Enables prediction of behavior at different load and slip conditions.
		\item Essential for designing protection and control systems.
	\end{itemize}
	
\end{document}
