\documentclass[12pt,a4paper]{article}
\usepackage[utf8]{inputenc}
\usepackage{enumitem}
\usepackage{geometry}
\usepackage{setspace}
\geometry{margin=1in}
\title{\textbf{Machine Viva and Lab Questions}}
\author{}
\date{}

\begin{document}
	
	\maketitle
	\onehalfspacing
	
	\section*{List of Questions}
	
	\begin{enumerate}
		\item Which condition cannot be checked using the 3-lamp method?
		\item How can we know that the phase sequence is the same?
		\item Why is the OCC characteristic has curve?
		\item Why is rated voltage used in the locked rotor test?
		\item What are the shortcomings of the 3-lamp method?
		\item Why is the capacitor curve upward in the synchronous generator’s load characteristics?
		\item In the determination of equivalent circuit parameters, what do Rc and Xm represent?
		\item Why does the OCC (open circuit characteristic) curve become saturated during the Xs determination?
		\item What type of load does a synchronous motor take?
		\item What should be done to operate a synchronous motor as a capacitive load?
		\item What is the significance of -Q in power systems?
		\item In the no-load test, which value is rated and why is it used?
		\item Same question for the locked rotor test.
		\item What does Rc represent in the equivalent circuit?
		\item What does Xm represent in the equivalent circuit?
		\item Why does the terminal voltage of a synchronous motor increase when operating with a capacitor load?
		\item In the 3-lamp method, which of the four criteria cannot be fully fulfilled and why?
		\textbf{answer :} This method connects three lamps across different phases between the incoming generator and the existing system. The brightness of the lamps indicates the phase difference: all dark means perfect phase match, one bright means a 120-degree phase difference, and two bright means a 240-degree phase difference. However, it doesn't provide precise information about the exact degree of phase difference within those ranges. 
		
		\item Which parameters cannot be detected using the 3-lamp method?
		\item How to increase active power in this context?
		\item How to increase reactive power?
		\item How much current flows in an induction motor during no-load? (30–60\% of rated)
		\item Which part of the torque-speed characteristics can be obtained from experiments?
		\item Why does the core saturate during the open-circuit test but remain linear during the short-circuit test?
		\item Does reactance decrease with increased field current?
		\item How to increase active and reactive power sharing of a generator on an infinite bus?
		\item How do you know when the phase angles are equal in the 3-lamp method?
		\item When does the synchronous machine operate at power factor 1?
		\item What cannot be measured using the 3-lamp method?
		\item How do you change power sharing in the 3-lamp method?
		\item In a DC test, which value is rated and why do we use it?
		\item In the no-load test, which value is rated and why?
		\item From which test do we get core losses? Which losses are summed in that test?
		\item Why is the open circuit curve not straight and the short circuit curve linear in reactance finding?
		\item In an induction motor, how does torque change with load and why?
		\item What happens to reactive power if active power increases?
		\item What happens to speed when load changes?
		\item Why does current increase or decrease when load is increased?
		\item Why do we use the linear part of the torque-speed curve for data?
		\item Does active and reactive power increase or decrease with load?
		\item Why isn't the core loss component present in the equivalent circuit of an induction motor?
		\item What happens to active and reactive power when load increases? Why?
		\item What is the meaning of -Q and +Q?
		\item How do we apply load to a synchronous motor or increase its load?
		\item In paralleling two alternators, which parameter cannot be perfectly achieved?
		\item How do you ensure both phase sequences are aligned?
		\item Why does load current increase with motor load?
		\item Why is reactive power in induction motor initially negative and then positive? What does this imply?
	\end{enumerate}
	
\end{document}
