\documentclass[a4paper,12pt]{article}

\usepackage{graphicx} % Required for inserting images
\usepackage{amsmath,amssymb,amsfonts}
\usepackage{subcaption}
% -----------------------
% Package Imports
% -----------------------

% Set page margins
\usepackage[a4paper, top=1in, bottom=0.8in, left=1.1in, right=0.8in]{geometry}

% Use Times New Roman font
\usepackage{times}

% Add page numbering
\pagestyle{plain}
\usepackage{multirow}
% Enable graphics inclusion
\usepackage{graphicx}
\usepackage{float}
% Enable code listings
\usepackage{listings}
\usepackage{xcolor} % For customizing code colors

% Define MATLAB style for listings
\lstdefinestyle{vscode-light}{
	language=Matlab,
	basicstyle=\ttfamily\footnotesize,
	keywordstyle=\color{blue},
	commentstyle=\color{gray},
	stringstyle=\color{red},
	numberstyle=\tiny\color{black},
	numbersep=5pt,
	frame=single,
	backgroundcolor=\color{white!10},
	breaklines=true,
	captionpos=b,
	tabsize=4,
	showstringspaces=false,
	numbers=left,  % Enable line numbering on the left
	stepnumber=1,  % Line numbers increment by 1
	numberfirstline=true, % Number the first line
}
\setlength{\parindent}{0pt}


\begin{document}
	\section{Experiment No. 4}
	
	\section{Experiment Title }
Solving Non-linear Equations Using Newton-Raphson and Secant Method
	\section{Objective}
	
	The objectives of this lab are:
	\begin{itemize}
	\item To understand and implement the Newton-Raphson and Secant methods for solving non-linear equations.
	\item To display the approximations, root and error for each iteration in a tabular form.
		
	\end{itemize}
	\section{Theory}
	
	\subsection{Newton-Raphson Method}
	The Newton-Raphson method is an iterative technique to find the roots of a real-valued function. It is based on the idea of linear approximation using tangents. Starting from an initial guess $x_0$, the next approximation is computed using the formula:
	
	\[
	x_{n+1} = x_n - \frac{f(x_n)}{f'(x_n)}
	\]
	
	The iteration continues until the error falls below a pre-defined tolerance.
	
	\subsubsection{Algorithm:}
	\begin{enumerate}
		\item Start with an initial guess $x_0$.
		\item Compute $x_{n+1} = x_n - \frac{f(x_n)}{f'(x_n)}$.
		\item Compute the error $|x_{n+1} - x_n|$.
		\item If error is less than tolerance, stop. Otherwise, update $x_n = x_{n+1}$ and repeat.
	\end{enumerate}
	
	\subsection{Secant Method}
	The Secant method is also an iterative root-finding algorithm but does not require the derivative of the function. It uses two initial approximations $x_0$ and $x_1$, and computes subsequent approximations using the formula:
	
	\[
	x_{n+1} = x_n - f(x_n) \cdot \frac{x_n - x_{n-1}}{f(x_n) - f(x_{n-1})}
	\]
	
	\subsubsection{Algorithm:}
	\begin{enumerate}
		\item Choose two initial values $x_0$ and $x_1$.
		\item Compute $x_{n+1}$ using the secant formula.
		\item Compute the error $|x_{n+1} - x_n|$.
		\item If error is below tolerance, stop. Otherwise, update $x_{n-1} = x_n$, $x_n = x_{n+1}$ and repeat.
	\end{enumerate}
	
	
	


	\subsection{Solving Non-linear Equation Using Newton-Raphson Method}
	Here $f(x) = x^2 - 2x - 5$, 
	error is assumed to be 0.001 
	
	
	
	\subsubsection{MATLAB Code:}
	\begin{lstlisting}[style=vscode-light, caption={Solving Non-linear Equation Using Newton-Raphson Method in MATLAB.} ]
		
clc;
clear;
	
f = @(x) x^2 - 2*x - 5;  
df = @(x) 2*x - 2;  
error = 0.001;   
N = 500;        
	
x0 = input('Enter initial value \n');  
	
for k = 1:N
if df(x0) == 0
fprintf('Not Converging, df(x0) is zero. Enter another initial value \n');
x0 = input('Enter another value: \n');
continue;  
end
	
x1 = x0 - f(x0) / df(x0);  
er = abs(x1 - x0);         
	
fprintf('Iteration %d: x0 = %.6f, x1 = %.6f, f(x1) = %.6f, error = %.6f\n', k, x0, x1, f(x1), er);
	
if abs(f(x1)) < error || er < error
fprintf('The root of the equation is %.6f\n', x1);
break;
end
	
x0 = x1;  
end
	
		
		
	\end{lstlisting}
	
	
	
	\newpage
	\subsection{Result Shown in Command Window}
	
	\begin{lstlisting}[style=vscode-light, caption={Command Window for Newton-Raphson Method} ]
Enter initial value 
1
Not Converging, df(x0) is zero. Enter another initial value 
Enter another value: 
2
Iteration 2: x0 = 2.000000, x1 = 4.500000,f(x1) = 6.250000, error = 2.500000
Iteration 3: x0 = 4.500000, x1 = 3.607143,f(x1) = 0.797194, error = 0.892857
Iteration 4: x0 = 3.607143, x1 = 3.454256,f(x1) = 0.023374, error = 0.152886
Iteration 5: x0 = 3.454256, x1 = 3.449494,f(x1) = 0.000023, error = 0.004762
The root of the equation is 3.449494
	\end{lstlisting}
	
	\subsection{Solving Non-linear Equation Using Secant Method}
	Here $f(x) = x^3 - 2x - 5$, 
	error is assumed to be 0.001 
	
	
	
	\subsubsection{MATLAB Code:}
	\begin{lstlisting}[style=vscode-light, caption={Solving Non-linear Equation Using  Secant Method in MATLAB.} ]
		
clc
clear

f = @(x) x^3 - 2*x -5;  
df = @(x) 3*x^2 - 2;  
error = 0.001;   
N = 500;        
x0 = 5;
x1 = 10;

for k = 1:N
x2 = x1 - f(x1)*(x1-x0) / (f(x1)-f(x0));  
er = abs(x2 - x1);        

fprintf('Iteration %d: x0 = %.6f, x1 = %.6f,f(x1) = %.6f,error = %.6f\n', k, x0, x1, f(x1), er);

if abs(f(x2)) < error || er < error
fprintf('The root of the equation is %.6f\n', x2);
break;

end

x0 = x1;  
x1= x2;
end


	
	
		
		
		
		
	\end{lstlisting}
	
	
	
	\newpage
	\subsection{Result Shown in Command Window}
	
	\begin{lstlisting}[style=vscode-light, caption={Command Window for Secant Method} ]
Iteration 1: x0 = 5.000000, x1 = 10.000000,f(x1) = 975.000000,error=5.635838
Iteration 2: x0 = 10.000000, x1 = 4.364162,f(x1) = 69.391104,error=0.431839
Iteration 3: x0 = 4.364162, x1 = 3.932323,f(x1) = 47.941514,error = 0.965193
Iteration 4: x0 = 3.932323, x1 = 2.967130,f(x1) = 15.187929,error = 0.447563
Iteration 5: x0 = 2.967130, x1 = 2.519567,f(x1) = 5.955621,error = 0.288716
Iteration 6: x0 = 2.519567, x1 = 2.230851,f(x1) = 1.640562,error = 0.109768
Iteration 7: x0 = 2.230851, x1 = 2.121082,f(x1) = 0.300566,error = 0.024621
Iteration 8: x0 = 2.121082, x1 = 2.096461,f(x1) = 0.021337,error = 0.001881
The root of the equation is 2.094580
		
	\end{lstlisting}
	
	
	
	
	
	\section{Discussion}
	In this experiment, the Newton-Raphson and Secant methods was introduced to solve a non-linear equation, highlighting its effectiveness in root-finding using MATLAB. 
	
	The Newton-Raphson Method demonstrated fast convergence when the derivative was not zero. It required a correct initial guess to avoid division by zero or divergence. .It was observed that this method guarantees finding only one root at a time. From the results, Using an error tolerance of 0.001, the root was found in 4 effective iterations after correcting the initial guess to converge, yielding a root of 3.449494 with an error of 0.004762. 
	
	The Secant Method, on the other hand, does not need the derivative of the function, which makes it suitable for functions that are hard to differentiate. Starting with two initial values, the method converged after 8 iterations and provided a reliable root estimate with acceptable error.It was also observed that this method guarantees finding only one root at a time. From the results, Using an error tolerance of 0.001, the root was found in 8 effective iterations after correcting the initial guess to converge, yielding a root of 2.094580 with an error of 0.001881.The number of iterations depends on the initial guess x0 and x1.
	
	Overall, both methods effectively found the roots, and the iteration results verified the accuracy and performance of each approach under the given stopping criteria.
.
	
	
	
	
	
	
	
	
\end{document}