\documentclass[a4paper,12pt]{article}

\usepackage{graphicx} % Required for inserting images
\usepackage{amsmath,amssymb,amsfonts}
\usepackage{subcaption}
% -----------------------
% Package Imports
% -----------------------

% Set page margins
\usepackage[a4paper, top=1in, bottom=0.8in, left=1.1in, right=0.8in]{geometry}

% Use Times New Roman font
\usepackage{times}

% Add page numbering
\pagestyle{plain}
\usepackage{multirow}
% Enable graphics inclusion
\usepackage{graphicx}
\usepackage{float}
% Enable code listings
\usepackage{listings}
\usepackage{xcolor} % For customizing code colors
% -----------------------
% Section Font Customization
% -----------------------
\usepackage{titlesec} % To customize section font size
\titleformat{\section}
{\normalfont\fontsize{14}{16}\bfseries}{\thesection}{1em}{}

\titleformat{\subsection}
{\normalfont\fontsize{14}{16}\bfseries}{\thesubsection}{1em}{}


% Define MATLAB style for listings
\lstdefinestyle{vscode-light}{
	language=Matlab,
	basicstyle=\ttfamily\footnotesize,
	keywordstyle=\color{black},
	commentstyle=\color{gray},
	stringstyle=\color{red},
	numberstyle=\tiny\color{black},
	numbersep=5pt,
	frame=single,
	backgroundcolor=\color{white!10},
	breaklines=true,
	captionpos=b,
	tabsize=4,
	showstringspaces=false,
	numbers=left,  % Enable line numbering on the left
	stepnumber=1,  % Line numbers increment by 1
	numberfirstline=true, % Number the first line
}
\setlength{\parindent}{0pt}


\begin{document}
	\section{Experiment No. 7}
	
	\section{Experiment Title }
	Solving system of linear equations and matrix inversion using LU Decomposition Method
	\section{Objective}
	
	The objectives of this lab are:
	\begin{itemize}
		\item To gather knowledge about solving a system of linear equations using the LU Decomposition Method.
		\item To implement and show output using the  LU Decomposition Method for solving linear equation and inverse matrix.
		
	\end{itemize}
	
\section{Theory}

The LU factorization method is a technique used to solve systems of linear algebraic equations of the form:
\begin{equation}
	AX = B \tag{7.1}
\end{equation}

Although it is a valid and sound method, solving the system repeatedly with the same coefficient matrix \( A \) but with different right-hand-side vectors \( B \) can be inefficient. LU decomposition addresses this by separating the time-consuming elimination of matrix \( A \) from the manipulation of the vector \( B \). Thus, once \( A \) is decomposed, multiple right-hand-side vectors can be processed efficiently.

The LU decomposition splits \( A \) into a lower triangular matrix \( L \) and an upper triangular matrix \( U \):
\begin{equation}
	A = LU \tag{7.2}
\end{equation}

Substituting into Eq.~(7.1), we get:
\begin{equation}
	LUX = B \tag{7.3}
\end{equation}

Let:
\begin{equation}
	UX = D \tag{7.4}
\end{equation}
Then,
\begin{equation}
	LD = B \tag{7.5}
\end{equation}

Thus, the LU factorization method consists of two main steps:

\begin{enumerate}
	\item \textbf{LU Decomposition Step:} The coefficient matrix \( A \) is decomposed into a lower triangular matrix \( L \) and an upper triangular matrix \( U \).
	\item \textbf{Substitution Step:} 
	\begin{enumerate}
		\item Forward substitution is applied to Eq.~(7.5) to solve for the intermediate vector \( D \).
		\item Back substitution is then used on Eq.~(7.4) to solve for \( X \).
	\end{enumerate}
\end{enumerate}

\subsection{LU Decomposition}

Gauss elimination can be used to decompose \( A \) into \( L \) and \( U \). Let:
\[
A = 
\begin{bmatrix}
	a_{11} & a_{12} & a_{13} \\
	a_{21} & a_{22} & a_{23} \\
	a_{31} & a_{32} & a_{33}
\end{bmatrix}
\]

The forward elimination process reduces \( A \) to:
\[
U = 
\begin{bmatrix}
	a_{11} & a_{12} & a_{13} \\
	0 & a'_{22} & a'_{23} \\
	0 & 0 & a''_{33}
\end{bmatrix}
\]

Meanwhile, the multipliers used in elimination form the matrix \( L \):
\[
L = 
\begin{bmatrix}
	1 & 0 & 0 \\
	L_{21} & 1 & 0 \\
	L_{31} & L_{32} & 1
\end{bmatrix}
\]
where:
\[
L_{21} = \frac{a_{21}}{a_{11}}, \quad
L_{31} = \frac{a_{31}}{a_{11}}, \quad
L_{32} = \frac{a'_{32}}{a'_{22}}
\]

\subsection{Finding {$X$} by Substitution}

After obtaining \( L \) and \( U \), we solve \( LD = B \) using forward substitution:
\[
\begin{bmatrix}
	1 & 0 & 0 \\
	l_{21} & 1 & 0 \\
	l_{31} & l_{32} & 1
\end{bmatrix}
\begin{bmatrix}
	d_1 \\ d_2 \\ d_3
\end{bmatrix}
=
\begin{bmatrix}
	b_1 \\ b_2 \\ b_3
\end{bmatrix}
\Rightarrow
\left\{
\begin{array}{l}
	d_1 = b_1 \\
	d_2 = b_2 - l_{21} d_1 \\
	d_3 = b_3 - l_{31} d_1 - l_{32} d_2
\end{array}
\right.
\]

Then we solve \( UX = D \) using back substitution:
\[
\begin{bmatrix}
	u_{11} & u_{12} & u_{13} \\
	0 & u_{22} & u_{23} \\
	0 & 0 & u_{33}
\end{bmatrix}
\begin{bmatrix}
	x_1 \\ x_2 \\ x_3
\end{bmatrix}
=
\begin{bmatrix}
	d_1 \\ d_2 \\ d_3
\end{bmatrix}
\Rightarrow
\left\{
\begin{array}{l}
	x_3 = \frac{d_3}{u_{33}} \\
	x_2 = \frac{d_2 - u_{23} x_3}{u_{22}} \\
	x_1 = \frac{d_1 - u_{12} x_2 - u_{13} x_3}{u_{11}}
\end{array}
\right.
\]

\subsection{Finding Inverse of a Matrix}

The inverse \( A^{-1} \) satisfies:
\begin{equation}
	AA^{-1} = I \tag{7.6}
\end{equation}

If \( AX = B \) and \( B = I \), then \( X = A^{-1} \). Using \( A = LU \) and substituting:
\[
A X = LU X = L D = B
\]

For a \( 3 \times 3 \) matrix:
\[
B = 
\begin{bmatrix}
	1 & 0 & 0 \\
	0 & 1 & 0 \\
	0 & 0 & 1
\end{bmatrix}
= [B_1\ B_2\ B_3]
\]

Steps to find \( A^{-1} \):

\begin{enumerate}
	\item For each column \( B_i \) of \( B \), solve \( L D_i = B_i \) by forward substitution.
	\item Then solve \( U X_i = D_i \) by back substitution.
	\item Repeat for \( i = 1, 2, 3 \), and form the inverse matrix:
	\[
	A^{-1} = [X_1\ X_2\ X_3]
	\]
\end{enumerate}

For example, to find \( D_1 \):
\[
\begin{bmatrix}
	1 & 0 & 0 \\
	l_{21} & 1 & 0 \\
	l_{31} & l_{32} & 1
\end{bmatrix}
\begin{bmatrix}
	d_{11} \\ d_{21} \\ d_{31}
\end{bmatrix}
=
\begin{bmatrix}
	b_{11} \\ b_{21} \\ b_{31}
\end{bmatrix}
\Rightarrow
\left\{
\begin{array}{l}
	d_{11} = b_{11} \\
	d_{21} = b_{21} - l_{21} d_{11} \\
	d_{31} = b_{31} - l_{31} d_{11} - l_{32} d_{21}
\end{array}
\right.
\]

Then solve:
\[
\begin{bmatrix}
	u_{11} & u_{12} & u_{13} \\
	0 & u_{22} & u_{23} \\
	0 & 0 & u_{33}
\end{bmatrix}
\begin{bmatrix}
	x_{11} \\ x_{21} \\ x_{31}
\end{bmatrix}
=
\begin{bmatrix}
	d_{11} \\ d_{21} \\ d_{31}
\end{bmatrix}
\Rightarrow
\left\{
\begin{array}{l}
	x_{31} = \frac{d_{31}}{u_{33}} \\
	x_{21} = \frac{d_{21} - u_{23} x_{31}}{u_{22}} \\
	x_{11} = \frac{d_{11} - u_{12} x_{21} - u_{13} x_{31}}{u_{11}}
\end{array}
\right.
\]

Repeat this process for all columns to obtain \( A^{-1} \).

\section{Algorithm}

\subsection{LU Decomposition}
\textbf{Input:} Square matrix $A$ of size $n \times n$ \\
\textbf{Output:} Lower triangular matrix $L$ and upper triangular matrix $U$
\begin{enumerate}
	\item Initialize $L$ as an identity matrix of size $n \times n$
	\item For $k = 1$ to $n - 1$:
	\begin{enumerate}
		\item For $i = k + 1$ to $n$:
		\begin{enumerate}
			\item $L[i,k] \leftarrow A[i,k] / A[k,k]$
			\item $A[i,:] \leftarrow A[i,:] - (A[i,k]/A[k,k]) \cdot A[k,:]$
		\end{enumerate}
	\end{enumerate}
	\item Set $U \leftarrow A$
	\item Return $L$ and $U$
\end{enumerate}

\subsection{Solving System Using LU Decomposition}
\textbf{Input:} Matrices $L$, $U$, and $B$ ($n \times 1$) \\
\textbf{Output:} Solution vector $X$

\textbf{Step 1: Forward Substitution (Find $D$)}
\begin{enumerate}
	\item Initialize $D$ and $D_{\text{sum}}$ as zero vectors
	\item For $j = 1$ to $n$:
	\begin{enumerate}
		\item $D_{\text{sum}}[j] \leftarrow \sum_{k=1}^{j-1} D[k] \cdot L[j,k]$
		\item $D[j] \leftarrow B[j] - D_{\text{sum}}[j]$
	\end{enumerate}
\end{enumerate}

\textbf{Step 2: Backward Substitution (Find $X$)}
\begin{enumerate}
	\item Initialize $X$ and $\text{sum}$ as zero vectors
	\item For $j = n$ down to $1$:
	\begin{enumerate}
		\item $\text{sum}[j] \leftarrow \sum_{k=j+1}^{n} U[j,k] \cdot X[k]$
		\item $X[j] \leftarrow (D[j] - \text{sum}[j]) / U[j,j]$
	\end{enumerate}
\end{enumerate}

\subsection{Matrix Inversion Using LU Decomposition}
\textbf{Input:} Matrices $L$, $U$, and identity matrix $B$ ($n \times n$) \\
\textbf{Output:} Inverse matrix $A^{-1}$

\textbf{Step 1: Forward Substitution (Find $D$)}
\begin{enumerate}
	\item Initialize $D$ and $D_{\text{sum}}$ as zero matrices
	\item For $i = 1$ to $n$:
	\begin{enumerate}
		\item For $j = 1$ to $n$:
		\begin{enumerate}
			\item $D_{\text{sum}}[j,i] \leftarrow \sum_{k=1}^{j-1} D[k,i] \cdot L[j,k]$
			\item $D[j,i] \leftarrow B[j,i] - D_{\text{sum}}[j,i]$
		\end{enumerate}
	\end{enumerate}
\end{enumerate}

\textbf{Step 2: Backward Substitution (Find $A^{-1}$)}
\begin{enumerate}
	\item Initialize $A^{-1}$ and sum as zero matrices
	\item For $i = 1$ to $n$:
	\begin{enumerate}
		\item For $j = n$ down to $1$:
		\begin{enumerate}
			\item $\text{sum}[j,i] \leftarrow \sum_{k=j+1}^{n} U[j,k] \cdot A^{-1}[k,i]$
			\item $A^{-1}[j,i] \leftarrow (D[j,i] - \text{sum}[j,i]) / U[j,j]$
		\end{enumerate}
	\end{enumerate}
\end{enumerate}

\subsection{Verification}
\textbf{Input:} $L$, $U$, $A^{-1}$, original matrix $A$ \\
\textbf{Steps:}
\begin{enumerate}
	\item Compute $L_d = L - \text{MATLAB's } L$
	\item Compute $U_d = U - \text{MATLAB's } U$
	\item Compute $A_d = A^{-1} - \text{MATLAB's } \text{inv}(A)$
	\item Verify if $L_d$, $U_d$, and $A_d$ are zero matrices
\end{enumerate}

	\newpage
	\subsection{Solving Non-linear Equation Using LU Decomposition Method}
	
	\subsubsection{MATLAB Code:}
	\begin{lstlisting}[style=vscode-light, caption={Solving Non-linear Equation Using LU Decomposition in MATLAB.} ]
		clc;
		clear;
		close all;
		
		A = [4, -2, 1; 20, -7, 12; -8, 13, 17];
		B = [11; 70; 17];
		n = size(A,1);
		L = zeros(n); 
		U = A;        
		
		for k = 1:n-1
		L(k,k) = 1; 
		for i = k+1:n
		L(i,k) = U(i,k) / U(k,k);
		U(i,:) = U(i,:) - L(i,k) * U(k,:);
		end
		end
		L(n,n) = 1; 
		LU = L * U;
		disp('Lower Triangular Matrix L:');
		disp(L);
		disp('Upper Triangular Matrix U:');
		disp(U);
		disp('Display the L*U')
		disp(LU)
		
		%for finding D
		D = zeros(n,1);
		for j = 1:n
		Dsum = 0;
		for k = 1:j-1
		Dsum = Dsum + D(k) * L(j,k);
		end
		D(j) = B(j) - Dsum;
		end
		
		%for finding X
		X = zeros(n,1);
		for j = n:-1:1
		sum = 0;
		for k = j+1:n
		sum = sum + U(j,k) * X(k);
		end
		X(j) = (D(j) - sum ) / U(j,j);
		end
		disp('Solution Vector X:');
		disp(X);
		%%%%%%%%%%%%%%%%%%%%%%%%%%%%
		n = size(L,1);
		I = eye(n);       % Identity matrix
		A_inv = zeros(n); % To store the inverse
			\end{lstlisting}
			\newpage
            \begin{lstlisting}[style=vscode-light, caption={Solving Non-linear Equation Using LU Decomposition in MATLAB.} ]
		%  to solve LY = I
		Y = zeros(n,n);
		for col = 1:n
		for j = 1:n
		sum1 = 0;
		for k = 1:j-1
		sum1 = sum1 + L(j,k) * Y(k,col);
		end
		Y(j,col) = (I(j,col) - sum1) / L(j,j);
		end
		end
		%  to solve UX = Y
		for col = 1:n
		for j = n:-1:1
		sum2 = 0;
		for k = j+1:n
		sum2 = sum2 + U(j,k) * A_inv(k,col);
		end
		A_inv(j,col) = (Y(j,col) - sum2) / U(j,j);
		end
		end
		disp('Inverse Matrix of A:');
		disp(A_inv);
		disp('Verification:');
		m=A*A_inv;
		disp(m);
		
		
		
	\end{lstlisting}
	
	
	
	
	\subsection{Result Shown in Command Window}
	
	\begin{lstlisting}[style=vscode-light, caption={Command Window for LU Decomposition} ]
		Lower Triangular Matrix L:
		1     0     0
		5     1     0
		-2     3     1
		Upper Triangular Matrix U:
		4    -2     1
		0     3     7
		0     0    -2

		Display the L*U
		4    -2     1
		20    -7    12
		-8    13    17
		Solution Vector X:
		1
		-2
		3
		Inverse Matrix of A:
		11.4583   -1.9583    0.7083
		18.1667   -3.1667    1.1667
		-8.5000    1.5000   -0.5000
	    Verification:
	     1.0000         0         0
	    0.0000    1.0000         0
	    0         0    1.0000
	\end{lstlisting}
	
	
	
	\section{Discussion}
	In this experiment, the LU Decomposition method was used to solve systems of linear equations inverse of square matrix. That method was implemented in MATLAB and tested on standard systems. Initially input A and B  were assumed, and Gauss eliminations method was used to decompose matrix A to lower and upper triangular matrix. Then Gauss eliminations was used to solve the linear equation and to find inverse of the matrix.
	
    Due to lower and upper triangular matrices, it was easier to solve linear equation as well as finding inverse matrix. The results from the method was verified with MATLAB’s built-in solver that found to be accurate.
	
	
	
	
	
	
	
	
\end{document}