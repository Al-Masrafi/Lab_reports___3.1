\documentclass[a4paper,12pt]{article}

\usepackage{graphicx} % Required for inserting images
\usepackage{amsmath,amssymb,amsfonts}
\usepackage{subcaption}
% -----------------------
% Package Imports
% -----------------------

% Set page margins
\usepackage[a4paper, top=1in, bottom=0.8in, left=1.1in, right=0.8in]{geometry}

% Use Times New Roman font
\usepackage{times}

% Add page numbering
\pagestyle{plain}
\usepackage{multirow}
% Enable graphics inclusion
\usepackage{graphicx}
\usepackage{float}
% Enable code listings
\usepackage{listings}
\usepackage{xcolor} % For customizing code colors
% -----------------------
% Section Font Customization
% -----------------------
\usepackage{titlesec} % To customize section font size
\titleformat{\section}
{\normalfont\fontsize{14}{16}\bfseries}{\thesection}{1em}{}

\titleformat{\subsection}
{\normalfont\fontsize{14}{16}\bfseries}{\thesubsection}{1em}{}


% Define MATLAB style for listings
\lstdefinestyle{vscode-light}{
	language=Matlab,
	basicstyle=\ttfamily\footnotesize,
	keywordstyle=\color{black},
	commentstyle=\color{gray},
	stringstyle=\color{red},
	numberstyle=\tiny\color{black},
	numbersep=5pt,
	frame=single,
	backgroundcolor=\color{white!10},
	breaklines=true,
	captionpos=b,
	tabsize=4,
	showstringspaces=false,
	numbers=left,  % Enable line numbering on the left
	stepnumber=1,  % Line numbers increment by 1
	numberfirstline=true, % Number the first line
}
\setlength{\parindent}{0pt}


\begin{document}
	\section{Experiment No. 8}
	
	\section{Experiment Title }
Finding intermediate points using Lagrange Interpolating Polynomial.
	\section{Objective}
	
	The objectives of this lab are:
	\begin{itemize}
		\item To know how to fit an $n^{th}$ order Lagrange interpolating polynomial through (n+1) data points..
		\item To find intermediate values in between tabulated data points using Lagrange interpolating polynomial through MATLAB programming.
		
	\end{itemize}
	
	\section{Lagrange Interpolation}
	
	Lagrange interpolation is a polynomial interpolation technique used to estimate intermediate data points within the range of a discrete set of known data points. The theory behind the Lagrange interpolation method is based on constructing a polynomial that passes through a given set of points $(x_i, y_i)$. This is especially useful in numerical analysis for interpolation.
	
	\subsection{ Definition}
	
	Given $n+1$ data points $(x_0, y_0), (x_1, y_1), \ldots, (x_n, y_n)$, the Lagrange interpolating polynomial $P(x)$ is expressed as:
	
	\begin{equation}
		P(x) = \sum_{i=0}^{n} y_i L_i(x)
		\label{eq:lagrange_poly}
	\end{equation}
	
	where $L_i(x)$ are the Lagrange basis polynomials defined as:
	
	\begin{equation}
		L_i(x) = \prod_{\substack{j=0 \\ j \ne i}}^{n} \frac{x - x_j}{x_i - x_j}
		\label{eq:lagrange_basis}
	\end{equation}
	
	\subsection{Example}
	
	Suppose three data points are given: $(x_0, y_0) = (1, 2)$, $(x_1, y_1) = (3, 4)$, and $(x_2, y_2) = (4, 3)$.
	
	\begin{enumerate}
		\item Compute the basis polynomials $L_0(x)$, $L_1(x)$, and $L_2(x)$:
		
		\begin{align*}
			L_0(x) &= \frac{(x - x_1)(x - x_2)}{(x_0 - x_1)(x_0 - x_2)} = \frac{(x - 3)(x - 4)}{(1 - 3)(1 - 4)} = \frac{(x - 3)(x - 4)}{6} \\
			L_1(x) &= \frac{(x - x_0)(x - x_2)}{(x_1 - x_0)(x_1 - x_2)} = \frac{(x - 1)(x - 4)}{(3 - 1)(3 - 4)} = -\frac{(x - 1)(x - 4)}{2} \\
			L_2(x) &= \frac{(x - x_0)(x - x_1)}{(x_2 - x_0)(x_2 - x_1)} = \frac{(x - 1)(x - 3)}{(4 - 1)(4 - 3)} = \frac{(x - 1)(x - 3)}{3}
		\end{align*}
		
		\item Form $P(x)$:
		
		\begin{equation}
			P(x) = y_0 L_0(x) + y_1 L_1(x) + y_2 L_2(x)
		\end{equation}
		
		\item Simplify $P(x)$ to get the final polynomial if required.
	\end{enumerate}
	
	\subsection{Algorithm}
	
	\noindent \textbf{Input:} Data points $x = [x_0, x_1, ..., x_n]$ and $y = [y_0, y_1, ..., y_n]$ \\
	\textbf{Output:} Simplified Lagrange interpolating polynomial $P(X)$ and its value at a given $X\_value$
	
	\begin{enumerate}
		\item Initialize polynomial $P \leftarrow 0$
		\item Define symbolic variable $X$
		\item Let $n \leftarrow$ length($x$)
		\item \textbf{For} $i = 1$ to $n$ \textbf{do}:
		\begin{enumerate}
			\item Initialize Lagrange basis polynomial $L_i \leftarrow 1$
			\item \textbf{For} $j = 1$ to $n$ \textbf{do}:
			\begin{itemize}
				\item \textbf{If} $i \ne j$ \textbf{then} \\
				Update $L_i \leftarrow L_i \times \frac{X - x[j]}{x[i] - x[j]}$
			\end{itemize}
			\item Update polynomial $P \leftarrow P + y[i] \times L_i$
		\end{enumerate}
		\item Simplify the polynomial: $P \leftarrow \text{simplify}(P)$
		\item Display \texttt{"Lagrange Interpolating Polynomial:"}, $P$
		\item \textbf{Input:} $X\_value$ (specific value to evaluate $P$)
		\item Compute $P\_value \leftarrow \text{double(substitute } P \text{ with } X = X\_value)$
	\end{enumerate}
	
	\newpage
	\subsection{Finding intermediate points using Lagrange Interpolating Polynomial}
	
	\subsubsection{MATLAB Code:}
	\begin{lstlisting}[style=vscode-light, caption={Lagrange Interpolating Polynomial in MATLAB.} ]
	x = [1, 4, 6];
	y = [0, log(4), log(6)];
	v = 2;
	n = 3;       
	L = zeros(1, n);     
	for i = 1:n
	Li = 1;         
	for j = 1:n
	if i ~= j
	Li = Li * ((v - x(j)) / (x(i) - x(j)));
	end
	end
	L(i) = Li;       
	end
	f = y(1) * L(1) + y(2) * L(2) + y(3) * L(3)  
	
	
	
	
	
	

	\end{lstlisting}
	
	
	
	
	\subsection{Result Shown in Command Window}
	
	\begin{lstlisting}[style=vscode-light, caption={Command Window for Finding intermediate points using Lagrange Interpolating Polynomial} ]
f = 0.5658
	\end{lstlisting}
	
	
	
\section{Discussion}
In this experiment, we successfully applied the Lagrange Interpolating Polynomial to estimate intermediate values between known data points. The Lagrange method constructs a polynomial that passes through all given data points, and it is particularly useful when only a few data points are available and an estimation is needed at a point not explicitly listed.

We selected three known points and used MATLAB to implement the Lagrange interpolation formula. The program correctly calculated the Lagrange basis polynomials and combined them to evaluate the function value at a specific point, $x = 2$. The result obtained was approximately $f = 0.5658$ for $\ln(2)$ and expected natural for $\ln(2) \approx 0.6931$ which aligns with the expected natural logarithmic value between $\ln(1) = 0$ and $\ln(4) \approx 1.3863$.



	
	
	
	
	
	
\end{document}