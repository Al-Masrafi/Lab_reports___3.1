\documentclass[12pt, a4paper]{article}

%---PACKAGES---
\usepackage[left=2.5cm, right=2.5cm, top=2.5cm, bottom=2.5cm]{geometry}
\usepackage{graphicx}
\usepackage{amsmath}
\usepackage{amssymb}
\usepackage{enumitem}
\usepackage{hyperref}
\hypersetup{
	colorlinks=true,
	linkcolor=blue,
	filecolor=magenta,      
	urlcolor=cyan,
	pdftitle={Power System I - All Class Tests},
}

%---DOCUMENT---
\begin{document}
	
	\begin{center}
		\huge \textbf{Power System I (EEE 3111) \\ Complete Class Test Collection}
	\end{center}
	\tableofcontents
	\newpage
	
	%================================
	% PART I: SECTION A TESTS
	%================================
	\part{Section A Class Tests}
	
	\section{Class Test - 1}
	\begin{tabular}{ll}
		\textbf{Course No.:} EEE 3111 & \textbf{Marks:} 20 \\
		\textbf{Course Title:} Power System I & \textbf{Time:} 15 minutes \\
	\end{tabular}
	\hrule
	\vspace{0.5cm}
	
	\begin{enumerate}[label=\textbf{\arabic*.}]
		\item Analyze the effect of Earth on the capacitance of a single-phase overhead transmission line by deriving the capacitance expression using the method of images, and compare it with the expression obtained without considering the Earth's influence. \hfill \textbf{[CO1, 20 Marks]}
	\end{enumerate}
	
	\begin{figure}[h!]
		\centering
		% NOTE: Replace with the actual image file for Section A, Class Test 1
%			\includegraphics[width=\textwidth]{path/to/section_a_ct1.png}
		\caption{Image of Section A, Class Test - 1 Paper.}
	\end{figure}
	
	\newpage
	
	\section{Class Test - 2}
	\begin{tabular}{ll}
		\textbf{Course No.:} EEE 3111 & \textbf{Marks:} 20 \\
		\textbf{Course Title:} Power System I & \textbf{Time:} 25 minutes \\
	\end{tabular}
	\hrule
	\vspace{0.5cm}
	
	\begin{enumerate}[label=\textbf{\arabic*.}]
		\item Derive the expression for the receiving-end power circle diagram from the basic transmission line equation. Based on the derived equation, draw a receiving-end power circle diagram. Then, explain how this diagram can be used to evaluate voltage regulation in a transmission line, and identify the key parameters that influence the shape and position of the power circle. \hfill \textbf{[CO1, 12 Marks]}
		
		\item A 275 kV transmission line has the following line constants:
		\[ A = 0.85 \angle 5^\circ, \quad B = 200 \angle 75^\circ \]
		\begin{enumerate}[label=\textbf{(\alph*)}]
			\item Determine the power at unity power factor that can be received if the voltage profile at each end is to be maintained at 275 kV.
			\item What type and rating of compensation equipment would be required if the load is 150 MW at unity power factor with the same voltage profile as in part (a)?
			\item With the load as in part (b), what would be the receiving-end voltage if the compensation equipment is not installed?
		\end{enumerate}
		\hfill \textbf{[CO1, 8 Marks]}
	\end{enumerate}
	
	\begin{figure}[h!]
		\centering
		% NOTE: Replace with the actual image file for Section A, Class Test 2
%			\includegraphics[width=\textwidth]{path/to/section_a_ct2.png}
		\caption{Image of Section A, Class Test - 2 Paper.}
	\end{figure}
	
	\newpage
	
	\section{Class Test - 3}
	\begin{tabular}{ll}
		\textbf{Course No.:} EEE 3111 & \textbf{Marks:} 20 \\
		\textbf{Course Title:} Power System I & \textbf{Time:} 25 minutes \\
	\end{tabular}
	\hrule
	\vspace{0.5cm}
	
	\begin{enumerate}[label=\textbf{\arabic*.}]
		\item Explain the physical mechanism of corona formation in high-voltage transmission line. Based on the explanation, identify and analyze the factors that affect its severity, and suggest suitable techniques to reduce its impact on high-voltage transmission line. \hfill \textbf{[CO1, 6+4+2 Marks]}
		
		\item A string of 4 insulators has a self-capacitance that is 10 times the pin-to-earth capacitance.
		\begin{enumerate}[label=\textbf{(\roman*)}]
			\item Determine the voltage distribution across each unit in the string, expressed as a percentage of the total voltage.
			\item Calculate the string efficiency.
		\end{enumerate}
		\hfill \textbf{[CO1, 6+2 Marks]}
	\end{enumerate}
	
	\begin{figure}[h!]
		\centering
		% NOTE: Replace with the actual image file for Section A, Class Test 3
%			\includegraphics[width=\textwidth]{path/to/section_a_ct3.png}
		\caption{Image of Section A, Class Test - 3 Paper.}
	\end{figure}
	
	\newpage
	
	\section{Class Test - 4}
	\begin{tabular}{ll}
		\textbf{Course No.:} EEE 3111 & \textbf{Marks:} 20 \\
		\textbf{Course Title:} Power System I & \textbf{Time:} 25 minutes \\
	\end{tabular}
	\hrule
	\vspace{0.5cm}
	
	\begin{enumerate}[label=\textbf{\arabic*.}]
		\item Explain why dielectric stress in the insulation of an underground cable is inherently non-uniform. Analyze how capacitance grading and intersheath grading contribute to a more uniform stress distribution, using appropriate diagrams and mathematical expressions. Identify the practical challenges involved in implementing these techniques in modern cable systems. \hfill \textbf{[CO3, 4+8+2 Marks]}
		
		\item A single-core cable is used on a 3-phase, 33 kV system with a supply frequency of 50 Hz. The cable is 5 km long, has a conductor diameter of 10 cm, and is surrounded by impregnated paper insulation of thickness 7 cm. The relative permittivity of the insulation is 4, and the specific resistance of the insulation is $5 \times 10^{14}$ $\Omega$-cm. Calculate the following:
		\begin{enumerate}[label=\textbf{(\roman*)}]
			\item The insulation resistance of the cable.
			\item The capacitance of the cable.
			\item The charging current per phase.
		\end{enumerate}
		\hfill \textbf{[CO3, 2+2+2 Marks]}
	\end{enumerate}
	
	\begin{figure}[h!]
		\centering
		% NOTE: Replace with the actual image file for Section A, Class Test 4
	%		\includegraphics[width=\textwidth]{path/to/section_a_ct4.png}
		\caption{Image of Section A, Class Test - 4 Paper.}
	\end{figure}
	\newpage
	
	%================================
	% PART II: SECTION B TESTS
	%================================
	\part{Section B Class Tests}
	
	\section{Class Test - 1}
	\begin{tabular}{ll}
		\textbf{Full Marks:} 20 & \textbf{Time:} 25 Minutes \\
		\multicolumn{2}{l}{\textbf{Course No.: EEE 3111, CT-1}} \\
	\end{tabular}
	\hrule
	\vspace{0.5cm}
	
	\begin{enumerate}[label=\textbf{Q.\arabic*}]
		\item Analyze the concept of inductance in transmission lines by answering the following:
		\begin{enumerate}[label=\textbf{(\alph*)}]
			\item \textbf{Inductance of a 3-Phase Line with Unsymmetrical Spacing:} \hfill \textbf{[10 Marks, CO1]}
			\begin{enumerate}[label=(\roman*)]
				\item Explain the concept of transposition in transmission lines and its role in inductance.
				\item Derive the formula for the inductance of a 3-phase overhead transmission line with unsymmetrical conductor spacing.
			\end{enumerate}
			\item Calculate the GMR of each conductor in following figure considering radius r of each individual strand. \hfill \textbf{[10 Marks, CO1]}
			\begin{figure}[h!]
				\centering
				% NOTE: Replace with the diagram from Section B, CT-1
	%			\includegraphics[width=0.5\textwidth]{path/to/section_b_ct1_diagram.png}
				\caption{Conductor configurations for Section B, Q.1(b).}
			\end{figure}
		\end{enumerate}
	\end{enumerate}
	
	\begin{figure}[h!]
		\centering
		% NOTE: Replace with the actual image file for Section B, Class Test 1
	%		\includegraphics[width=\textwidth]{path/to/section_b_ct1.png}
		\caption{Image of Section B, Class Test - 1 Paper.}
	\end{figure}
	
	\newpage
	
	\section{Class Test - 3}
	\begin{tabular}{ll}
		\textbf{Full Marks:} 20 & \textbf{Time:} 25 Minutes \\
		\multicolumn{2}{l}{\textbf{Course No.: EEE 3111, CT-3}} \\
	\end{tabular}
	\hrule
	\vspace{0.5cm}
	
	\begin{enumerate}[label=\textbf{Q.\arabic*}]
		\item Each line of a 3-phase system is suspended by a string of 3 identical insulators of self-capacitance C farad. The shunt capacitance of connecting metal work of each insulator is 0.2 C to earth and 0.1 C to line. Calculate the string efficiency of the system if a guard ring increases the capacitance to the line of metal work of the lowest insulator to 0.3 C. \hfill \textbf{[10 Marks, CO3]}
		
		\item A transmission line has a span of 200 meters between level supports. The conductor has a cross-sectional area of 1.29 cm$^2$, weighs 1170 kg/km and has a breaking stress of 4218 kg/cm$^2$. Calculate the sag for a safety factor of 5, allowing a wind pressure of 122 kg per square meter of projected area. What is the vertical sag? \hfill \textbf{[10 Marks, CO3]}
	\end{enumerate}
	
	\begin{figure}[h!]
		\centering
		% NOTE: Replace with the actual image file for Section B, Class Test 3
%			\includegraphics[width=\textwidth]{path/to/section_b_ct3.png}
		\caption{Image of Section B, Class Test - 3 Paper.}
	\end{figure}
	
	\newpage
	
	\section{Class Test - 4}
	\begin{tabular}{ll}
		\textbf{Full Marks:} 20 & \textbf{Time:} 25 Minutes \\
		\multicolumn{2}{l}{\textbf{Course No.: EEE 3111, CT-4}} \\
	\end{tabular}
	\hrule
	\vspace{0.5cm}
	
	\begin{enumerate}[label=\textbf{Q.\arabic*}]
		\item A single core cable for use on 11 kV, 50 Hz system has conductor area of 0.645 cm$^2$ and internal diameter of sheath is 2.18 cm. The permittivity of the dielectric used in the cable is 3.5. Find (i) the maximum electrostatic stress in the cable (ii) minimum electrostatic stress in the cable (iii) capacitance of the cable per km length (iv) charging current. \hfill \textbf{[10 Marks, CO3]}
		
		\item A single core lead sheathed cable has a conductor diameter of 3 cm; the diameter of the cable being 9 cm. The cable is graded by using two dielectrics of relative permittivity 5 and 4 respectively with corresponding safe working stresses of 30 kV/cm and 20 kV/cm. Calculate the radial thickness of each insulation and the safe working voltage of the cable. \hfill \textbf{[10 Marks, CO3]}
	\end{enumerate}
	
	\begin{figure}[h!]
		\centering
		% NOTE: Replace with the actual image file for Section B, Class Test 4
%			\includegraphics[width=\textwidth]{path/to/section_b_ct4.png}
		\caption{Image of Section B, Class Test - 4 Paper.}
	\end{figure}
	\newpage
	
	%================================
	% PART III: SECTION C TESTS
	%================================
	\part{Section C Class Tests}
	\section{Class Test - 1}
	\begin{tabular}{ll}
		\textbf{Course No.:} EEE 3111 & \textbf{Marks:} 20 \\
		\textbf{Course Title:} Power System I & \textbf{Time:} 25 minutes \\
	\end{tabular}
	\hrule
	\vspace{0.5cm}
	
	\begin{enumerate}[label=\textbf{\arabic*.}]
		\item A single-phase transmission line consists of two composite conductors, designated as X and Y. Conductor X comprises $n$ strands, while conductor Y comprises $m$ strands. All strands are assumed to have the same radius. Derive a general expression for the inductance of the line. \hfill \textbf{[CO2, 14 Marks]}
		\item A single-phase line has two parallel conductors 2 m apart. The diameter of each conductor is 1.2 cm. Calculate the loop inductance per km of the line. \hfill \textbf{[CO2, 06 Marks]}
	\end{enumerate}
	
	\begin{figure}[h!]
		\centering
		% NOTE: Replace with the actual image file for Section C, Class Test 1
%			\includegraphics[width=\textwidth]{path/to/section_c_ct1.png}
		\caption{Image of Section C, Class Test - 1 Paper.}
	\end{figure}
	
	\newpage
	\section{Class Test - 2}
	\begin{tabular}{ll}
		\textbf{Course No.:} EEE 3111 & \textbf{Marks:} 20 \\
		\textbf{Course Title:} Power System I & \textbf{Time:} 25 minutes \\
	\end{tabular}
	\hrule
	\vspace{0.5cm}
	\begin{enumerate}[label=\textbf{\arabic*.}]
		\item Apply the distributed parameter model to derive the expressions for voltage and current as functions of distance along a long transmission line. Clearly demonstrate each step of the derivation process and explicitly state all underlying assumptions. \hfill \textbf{[CO2, 12 Marks]}
		\item An overhead 3-phase transmission line delivers 5000 kW at 22 kV at 0.8 p.f. lagging. The resistance and reactance of each conductor is 4 $\Omega$ and 6 $\Omega$ respectively. Determine: (i) sending end voltage (ii) voltage regulation, (iii) sending end power factor and (iv) transmission efficiency. \hfill \textbf{[CO2, 08 Marks]}
	\end{enumerate}
	\begin{figure}[h!]
		\centering
		% NOTE: Replace with the actual image file for Section C, Class Test 2
%			\includegraphics[width=\textwidth]{path/to/section_c_ct2.png}
		\caption{Image of Section C, Class Test - 2 Paper.}
	\end{figure}
	
	\newpage
	\section{Class Test - 3}
	\begin{tabular}{ll}
		\multicolumn{2}{l}{\textbf{Rajshahi University of Engineering \& Technology}} \\
		\multicolumn{2}{l}{\textbf{Department of Electrical and Electronic Engineering}} \\
		\textbf{Full Marks:} 20 & \textbf{Time:} 25 Minutes \\
		\multicolumn{2}{l}{\textbf{Course No.: EEE 3111, CT-3}} \\
	\end{tabular}
	\hrule
	\vspace{0.5cm}
	\begin{enumerate}[label=\textbf{Q.\arabic*}]
		\item Each line of a 3-phase system is suspended by a string of 3 identical insulators of self-capacitance C farad. The shunt capacitance of connecting metal work of each insulator is 0.2 C to earth and 0.1 C to line. Calculate the string efficiency of the system if a guard ring increases the capacitance to the line of metal work of the lowest insulator to 0.3 C. \hfill \textbf{[10 Marks, CO3]}
		
		\item A transmission line has a span of 200 meters between level supports. The conductor has a cross-sectional area of 1.29 cm$^2$, weighs 1170 kg/km and has a breaking stress of 4218 kg/cm$^2$. Calculate the sag for a safety factor of 5, allowing a wind pressure of 122 kg per square meter of projected area. What is the vertical sag? \hfill \textbf{[10 Marks, CO3]}
	\end{enumerate}
	\begin{figure}[h!]
		\centering
		% NOTE: Replace with the actual image file for Section C, Class Test 3
%			\includegraphics[width=\textwidth]{path/to/section_c_ct3.png}
		\caption{Image of Section C, Class Test - 3 Paper.}
	\end{figure}
	
	\newpage
	\section{Class Test - 4}
	\begin{tabular}{ll}
		\multicolumn{2}{l}{\textbf{Department of Electrical and Electronic Engineering}} \\
		\textbf{Full Marks:} 20 & \textbf{Time:} 25 Minutes \\
		\multicolumn{2}{l}{\textbf{Course No.: EEE 3111, CT-4}} \\
	\end{tabular}
	\hrule
	\vspace{0.5cm}
	\begin{enumerate}[label=\textbf{Q.\arabic*}]
		\item A single core cable for use on 11 kV, 50 Hz system has conductor area of 0.645 cm$^2$ and internal diameter of sheath is 2.18 cm. The permittivity of the dielectric used in the cable is 3.5. Find (i) the maximum electrostatic stress in the cable (ii) minimum electrostatic stress in the cable (iii) capacitance of the cable per km length (iv) charging current. \hfill \textbf{[10 Marks, CO3]}
		
		\item A single core lead sheathed cable has a conductor diameter of 3 cm; the diameter of the cable being 9 cm. The cable is graded by using two dielectrics of relative permittivity 5 and 4 respectively with corresponding safe working stresses of 30 kV/cm and 20 kV/cm. Calculate the radial thickness of each insulation and the safe working voltage of the cable. \hfill \textbf{[10 Marks, CO3]}
	\end{enumerate}
	\begin{figure}[h!]
		\centering
		% NOTE: Replace with the actual image file for Section C, Class Test 4
%		\includegraphics[width=\textwidth]{path/to/section_c_ct4.png}
		\caption{Image of Section C, Class Test - 4 Paper.}
	\end{figure}
	
\end{document}