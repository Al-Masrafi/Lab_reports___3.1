\documentclass[12pt, a4paper]{article}

%---PACKAGES---
\usepackage[left=2.5cm, right=2.5cm, top=2.5cm, bottom=2.5cm]{geometry}
\usepackage{graphicx}
\usepackage{amsmath}
\usepackage{amsfonts}
\usepackage{amssymb}
\usepackage{enumitem}
\usepackage{hyperref}
\hypersetup{
	colorlinks=true,
	linkcolor=blue,
	filecolor=magenta,      
	urlcolor=cyan,
}

%---DOCUMENT---
\title{Power System II - Questions Sorted by Topic}
\author{Exam Years: 2016-2021}
\date{\today}

\begin{document}
	\maketitle
	\tableofcontents
	\newpage
	
	\section{Power System Stability}
	
	%---2021---
	\subsection{Year 2021}
	\begin{enumerate}[label=\textbf{Q\arabic*.}, wide, labelindent=0pt]
		\item 
		\begin{enumerate}[label=\textbf{(\alph*)}]
			\item Conclude if the increases in the number of coherent machines improve the synchronous stability. Comment based on the swing equation of n-number of coherent machines.
			\item Analyze how a synchronizing power coefficient can define the nature of damping of the rotor angle.
			\item Outline the algorithm to find out the swing equation of a multi-machine power system during fault and after clearing the fault, where the line and transformer data, bus data, prefault load flow data and required generator data are given.
		\end{enumerate}
		\item 
		\begin{enumerate}[label=\textbf{(\alph*)}]
			\item State some measures to improve the synchronous stability of a power system.
			\item Demonstrate the equal area criteria of synchronous stability and use the criteria to solve for the critical clearing angle.
			\item A power system is shown below having both terminal voltage and infinite bus voltage of 1 per unit. The delivered power is 0.8 per unit. If a three phase fault occurs on the power system at the transmission line as indicated in the following figure. Determine (i) the power angle equation and (ii) the swing equation during fault. Let H = 5.00 MJ/MVA.
			\begin{figure}[h!]
				\centering
				% NOTE: Replace with your actual image file. Found on Page 1.
			%	\includegraphics[width=0.7\textwidth]{path/to/2021_q2c_diagram.png}
				\caption{Diagram for 2021, Q.2(c).}
			\end{figure}
		\end{enumerate}
	\end{enumerate}
	
	%---2020---
	\subsection{Year 2020}
	\begin{enumerate}[label=\textbf{Q\arabic*.}, wide, labelindent=0pt]
		\item
		\begin{enumerate}[label=\textbf{(\alph*)}]
			\item Explain the different steps of a "step by step approach" of solving the swing equation.
			\item The single-line diagram of a power system is shown below...Compute the power angle equation and the swing equation for the system under (i) pre-fault condition, (ii) during fault condition, and (iii) post-fault condition.
			\begin{figure}[h!]
				\centering
				% NOTE: Replace with your actual image file. Found on Page 3.
		%			\includegraphics[width=0.8\textwidth]{path/to/2020_q1b_diagram.png}
				\caption{Diagram for 2020, Q.1(b).}
			\end{figure}
			\item A 50 Hz, 4 pole turbogenerator rated 100 MVA, 13.8 KV has an inertia constant of 10 MJ/MVA. (i) Find the stored energy in the rotor... (ii) ...find the rotor acceleration.
		\end{enumerate}
		\item
		\begin{enumerate}[label=\textbf{(\alph*)}]
			\item Derive an expression of the critical clearing angle for the power system as shown in the following figure...
			\begin{figure}[h!]
				\centering
				% NOTE: Replace with your actual image file. Found on Page 3.
		%			\includegraphics[width=0.8\textwidth]{path/to/2020_q2a_diagram.png}
				\caption{Diagram for 2020, Q.2(a).}
			\end{figure}
			\item What assumptions need to be made during transient stability studies? What are the drawbacks of an equal-area criterion method?
			\item A loss-free generator supplies 50 MW to an infinite bus... Determine whether the generator will remain in synchronism if the prime mover input is abruptly increased by 30 MW.
		\end{enumerate}
	\end{enumerate}
	
	%---2019---
	\subsection{Year 2019}
	\begin{enumerate}[label=\textbf{Q.6}, wide, labelindent=0pt]
		\item 
		\begin{enumerate}[label=\textbf{(\alph*)}]
			\item Define (i) transient stability, (ii) dynamic stability, and (iii) steady-state stability with examples.
		\end{enumerate}
	\end{enumerate}
	\begin{enumerate}[label=\textbf{Q.7}, wide, labelindent=0pt, start=7]
		\item 
		\begin{enumerate}[label=\textbf{(\alph*)}]
			\item Single line diagram of the following figure shows a generator connected to an infinite bus... Determine the power angle equation for the system with the fault and the corresponding swing equation.
			\begin{figure}[h!]
				\centering
				% NOTE: Replace with your actual image file. Found on Page 6.
		%			\includegraphics[width=0.7\textwidth]{path/to/2019_q7c_diagram.png}
				\caption{Diagram for 2019, Q.7(c).}
			\end{figure}
		\end{enumerate}
	\end{enumerate}
	
	
	%---2018---
	\subsection{Year 2018}
	\begin{enumerate}[label=\textbf{Q\arabic*.}, wide, labelindent=0pt, start=5]
		\item
		\begin{enumerate}[label=\textbf{(\alph*)}]
			\setcounter{enumii}{3} % Start sub-enumeration at (d)
			\item What are the significances of the swing curve of the machine?
		\end{enumerate}
		\item
		\begin{enumerate}[label=\textbf{(\alph*)}]
			\item What do you mean by coherent machines? For any pair of non-coherent machine in a power system derive the swing equation.
		\end{enumerate}
		\item
		\begin{enumerate}[label=\textbf{(\alph*)}]
			\item What is dynamic stability? How does it differ from that of steady state stability and transient stability?
			\item Describe equal area criteria for power system stability and hence derive the equation of critical clearing angle.
			\item A 50 Hz synchronous generator... is connected to an infinite bus... Determine the critical clearing angle and the critical fault clearing time.
			\begin{figure}[h!]
				\centering
				% NOTE: Replace with your actual image file. Found on Page 8.
		%			\includegraphics[width=0.7\textwidth]{path/to/2018_q7c_diagram.png}
				\caption{Diagram for 2018, Q.7(c).}
			\end{figure}
		\end{enumerate}
	\end{enumerate}
	
	\newpage
	\section{Transmission Lines (Overhead)}
	
	%---2021---
	\subsection{Year 2021}
	\begin{enumerate}[label=\textbf{Q\arabic*.}, wide, labelindent=0pt, start=5]
		\item
		\begin{enumerate}[label=\textbf{(\alph*)}]
			\item Write down short notes on, (i) guard ring, (ii) critical disruptive voltage, and (iii) sag.
			\item Discuss the causes of insulator failure in overhead transmission lines with necessary sketches.
			\item An overhead transmission line at a river crossing is supported from two towers at heights of 40 m and 90 m above water level... calculate the clearance between the conductor and water at a point mid-way between the towers.
		\end{enumerate}
	\end{enumerate}
	
	%---2020---
	\subsection{Year 2020}
	\begin{enumerate}[label=\textbf{Q\arabic*.}, wide, labelindent=0pt, start=6]
		\item 
		\begin{enumerate}[label=\textbf{(\alph*)}]
			\setcounter{enumii}{2} % start sub-enumeration at (c)
			\item In a 33 KV overhead line, there are three units in the string of insulators...find (i) the distribution of voltage over three insulators, and (ii) string efficiency.
		\end{enumerate}
		\item
		\begin{enumerate}[label=\textbf{(\alph*)}]
			\setcounter{enumii}{2} % start sub-enumeration at (c)
			\item A 3-$\Phi$ overhead line has conductors, 2 cm in diameter spaced equilaterally 1 m apart... find the disruptive critical voltage for the line.
		\end{enumerate}
	\end{enumerate}
	
	\subsection{Year 2019}
	\begin{enumerate}[label=\textbf{Q\arabic*.}, wide, labelindent=0pt, start=1]
		\item 
		\begin{enumerate}[label=\textbf{(\alph*)}]
			\item "String efficiency for a DC system is 100\%" - prove this statement. Is it also true for an AC system?
			\item A string of 4 insulators has a self-capacitance equal to ten times the pin to earth capacitance. Find (i) the voltage distribution across various units... and (ii) string efficiency.
		\end{enumerate}
		\item 
		\begin{enumerate}[label=\textbf{(\alph*)}]
			\setcounter{enumii}{2} % start sub-enumeration at (c)
			\item A transmission line has a span of 275 m between level supports... calculate sag for a safety factor of 2.
		\end{enumerate}
	\end{enumerate}
	
	%---2017---
	\subsection{Year 2017}
	\begin{enumerate}[label=\textbf{Q.3}, wide, labelindent=0pt, start=3]
		\item 
		\begin{enumerate}[label=\textbf{(\alph*)}]
			\setcounter{enumii}{2} % start sub-enumeration at (c)
			\item Find the inductance per phase per km of double circuit 3-phase line shown in the figure.
			\begin{figure}[h!]
				\centering
				% NOTE: Replace with your actual image file. Found on Page 9.
		%			\includegraphics[width=0.6\textwidth]{path/to/2017_q3c_diagram.png}
				\caption{Diagram for 2017, Q.3(c).}
			\end{figure}
		\end{enumerate}
	\end{enumerate}
	
	
	\newpage
	\section{Distribution Systems (AC \& DC)}
	\subsection{Year 2021}
	\begin{enumerate}[label=\textbf{Q\arabic*.}, wide, labelindent=0pt, start=7]
		\item
		\begin{enumerate}[label=\textbf{(\alph*)}]
			\item Discuss the usage areas of DC and AC distribution. Write down short notes on, (i) feeder, (ii) bus bar and (iii) distributor.
			\item A d.c. 2-wire distributor AB is 500 m long and is fed at both ends at 240 V... Calculate (i) the point of minimum voltage and (ii) the value of this voltage.
			\begin{figure}[h!]
				\centering
				% NOTE: Replace with your actual image file. Found on Page 2.
			%	\includegraphics[width=0.8\textwidth]{path/to/2021_q7b_diagram.png}
				\caption{Diagram for 2021, Q.7(b).}
			\end{figure}
			\item Discuss the function of balancers in a 3-wire DC system. Show that a 3-wire system requires only 31.25\% as much copper as a 2-wire system.
		\end{enumerate}
	\end{enumerate}
	
	\subsection{Year 2019}
	\begin{enumerate}[label=\textbf{Q\arabic*.}, wide, labelindent=0pt, start=4]
		\item 
		\begin{enumerate}[label=\textbf{(\alph*)}]
			\setcounter{enumii}{2} % start sub-enumeration at (c)
			\item A 2-wire DC distributor AB is 300 m long. It is fed at point A. The various loads and their positions are given below... find the cross sectional area of the distributor.
			\begin{figure}[h!]
				\centering
				% NOTE: Replace with your actual image file. Found on Page 6.
			%	\includegraphics[width=0.8\textwidth]{path/to/2019_q4c_table.png}
				\caption{Table for 2019, Q.4(c).}
			\end{figure}
		\end{enumerate}
	\end{enumerate}
	
	
\end{document}