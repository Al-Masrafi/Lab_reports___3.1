\documentclass[12pt, a4paper]{article}

%---PACKAGES---
\usepackage[left=2cm, right=2cm, top=2.5cm, bottom=2.5cm]{geometry}
\usepackage{graphicx}
\usepackage{amsmath}
\usepackage{amsfonts}
\usepackage{amssymb}
\usepackage{enumitem}
\usepackage{hyperref}
\hypersetup{
	colorlinks=true,
	linkcolor=blue,
	filecolor=magenta,      
	urlcolor=cyan,
}

%---DOCUMENT---
\title{EEE Power Systems I \& II - Collected Questions}
\author{Sorted by Topic and Year}
\date{\today}

\begin{document}
	\maketitle
	\tableofcontents
	\newpage
	
	%========================================
	% Power System I Topics
	%========================================
	
	\part{Power System I (EEE 3211)}
	
	\section{Per-Unit System \& Impedance Diagrams}
	
	\subsection{Year 2021}
	\begin{enumerate}[label=\textbf{Q\arabic*.}, wide, labelindent=0pt]
		\item
		\begin{enumerate}[label=\textbf{(\alph*)}]
			\item What is the per-unit system and single line diagram of a power system? Prove that per-unit impedance of a transformer is the same whether computed from primary or secondary side.
			\item Explain mathematically how an existing power system network's Y-bus matrix gets affected when transmission is added to the network.
			\item Draw an impedance diagram for the following network showing in per-unit on a 100MVA base. Choose 20kV as the voltage base for generator. Three phase power and line-line ratings are given.
			\begin{figure}[h!]
				\centering
				% NOTE: Replace with image from 2021 (EEE 3211), Page 1.
		%			\includegraphics[width=\textwidth]{path/to/2021_eee3211_q1c.png}
				\caption{Network for 2021, Q.1(c).}
			\end{figure}
		\end{enumerate}
	\end{enumerate}
	
	\subsection{Year 2020}
	\begin{enumerate}[label=\textbf{Q.5.(c)}, wide, labelindent=0pt]
		\item Find the fault current in p.u. and amperes for the following network when a 3LG fault of zero impedance occurs at (i) point A and (ii) point B. Choose a base power of 100 MVA.
		\begin{figure}[h!]
			\centering
			% NOTE: Replace with image from 2020 (EEE 3211), Page 2 (labeled page 2/3).
		%		\includegraphics[width=\textwidth]{path/to/2020_eee3211_q5c.png}
			\caption{Network for 2020, Q.5(c).}
		\end{figure}
	\end{enumerate}
	
	\subsection{Year 2019}
	\begin{enumerate}[label=\textbf{Q\arabic*.}, wide, labelindent=0pt]
		\item 
		\begin{enumerate}[label=\textbf{(\alph*)}]
			\setcounter{enumii}{3} % Start letter at (d)
			\item Draw the impedance diagram for the power system which is shown in the following figure. The ratings of the generators, motors, and transformers are given.
			\begin{figure}[h!]
				\centering
				% NOTE: Replace with image from 2019 (EEE 3211), Page 1.
			%		\includegraphics[width=0.9\textwidth]{path/to/2019_eee3211_q1d.png}
				\caption{Network for 2019, Q.1(d).}
			\end{figure}
		\end{enumerate}
		\item 
		\begin{enumerate}[label=\textbf{(\alph*)}]
			\setcounter{enumii}{2} % Start letter at (c)
			\item Draw the impedance diagram and find the power consumption by the load for the electric power system as shown.
			\begin{figure}[h!]
				\centering
				% NOTE: Replace with image from 2019 (EEE 3211), Page 1.
			%		\includegraphics[width=0.9\textwidth]{path/to/2019_eee3211_q3c.png}
				\caption{Network for 2019, Q.3(c).}
			\end{figure}
		\end{enumerate}
	\end{enumerate}
	
	\section{Transmission Line Parameters \& Performance}
	
	\subsection{Year 2021}
	\begin{enumerate}[label=\textbf{Q\arabic*.}, wide, labelindent=0pt, start=2]
		\item
		\begin{enumerate}[label=\textbf{(\alph*)}]
			\setcounter{enumii}{2} % Start letter at (c)
			\item Prove that, the inductance per loop meter of a single-phase two wire line is $L = 4 \times 10^{-7} \ln \frac{D}{r'} \; H/m$.
		\end{enumerate}
	\end{enumerate}
	\begin{enumerate}[label=\textbf{Q\arabic*.}, wide, labelindent=0pt, start=3]
		\item
		\begin{enumerate}[label=\textbf{(\alph*)}]
			\item Calculate the capacitance of a three phase line with unsymmetrical spacing which is shown in fig. 3a.
			\begin{figure}[h!]
				\centering
				% NOTE: Replace with image from 2021 (EEE 3211), Page 2.
		%			\includegraphics[width=0.5\textwidth]{path/to/2021_eee3211_q3a.png}
				\caption{Line configuration for 2021, Q.3(a).}
			\end{figure}
			\item For medium transmission lines, prove that A = D and AD - BC = 1. Where symbols have their usual meanings. Consider nominal T-method.
			\item A 100km long, three phase, 50Hz transmission line has given constants. If the line supplies a load of 20MW, calculate by nominal $\pi$-method: (i) Sending end power factor, (ii) Regulation and, (iii) Transmission efficiency.
		\end{enumerate}
	\end{enumerate}
	
	
	\subsection{Year 2020}
	\begin{enumerate}[label=\textbf{Q\arabic*.}, wide, labelindent=0pt, start=2]
		\item 
		\begin{enumerate}[label=\textbf{(\alph*)}]
			\setcounter{enumii}{2} % Start letter at (c)
			\item One circuit of a 1-$\phi$ transmission line is composed of three solid 0.30 cm radius wires. The return circuit is composed of two 0.60 cm radius wires. The arrangement is shown. Find the inductance.
			\begin{figure}[h!]
				\centering
				% NOTE: Replace with image from 2020 (EEE 3211), Page 1.
		%			\includegraphics[width=0.8\textwidth]{path/to/2020_eee3211_q2c.png}
				\caption{Line configuration for 2020, Q.2(c).}
			\end{figure}
		\end{enumerate}
	\end{enumerate}
	
	\section{Load Flow Analysis}
	
	\subsection{Year 2021}
	\begin{enumerate}[label=\textbf{Q\arabic*.}, wide, labelindent=0pt, start=2]
		\item
		\begin{enumerate}[label=\textbf{(\alph*)}]
			\item How do we control active power flow and reactive power flow by tap changer of a transformer in a power system network?
			\item A sample power system network is shown in fig 2b. Construct the bus admittance matrix. The per-unit reactances are presented on the same base.
			\begin{figure}[h!]
				\centering
				% NOTE: Replace with image from 2021 (EEE 3211), Page 1.
		%			\includegraphics[width=0.8\textwidth]{path/to/2021_eee3211_q2b.png}
				\caption{Network for 2021, Q.2(b).}
			\end{figure}
		\end{enumerate}
	\end{enumerate}
	
	\begin{enumerate}[label=\textbf{Q\arabic*.}, wide, labelindent=0pt, start=4]
		\item 
		\begin{enumerate}[label=\textbf{(\alph*)}]
			\setcounter{enumii}{2} % Start letter at (c)
			\item For the following power system network... (i) Using Gauss-Seidal method, determine $V_2$ after two iterations. (ii) If after several iterations voltage at bus 2 converges to $V_2 = 0.90 - j0.10$, determine $S_1$ and real and reactive power losses in line.
			\begin{figure}[h!]
				\centering
				% NOTE: Replace with image from 2021 (EEE 3211), Page 2.
		%			\includegraphics[width=0.7\textwidth]{path/to/2021_eee3211_q4c.png}
				\caption{Network for 2021, Q.4(c).}
			\end{figure}
		\end{enumerate}
	\end{enumerate}
	
	\subsection{Year 2020}
	\begin{enumerate}[label=\textbf{Q.3(c)}, wide, labelindent=0pt]
		\item The one-line diagram of a simple three-bus power system is shown... (i) compute the bus voltages using Newton-Raphson method with initial estimates... and (ii) compute the slack bus real and reactive power.
		\begin{figure}[h!]
			\centering
			% NOTE: Replace with image from 2020 (EEE 3211), Page 1.
		%		\includegraphics[width=0.8\textwidth]{path/to/2020_eee3211_q3c.png}
			\caption{Network for 2020, Q.3(c).}
		\end{figure}
	\end{enumerate}
	
	\begin{enumerate}[label=\textbf{Q.4(c)}, wide, labelindent=0pt]
		\item Following figure shows the single line diagram of a simple power system... (i) Using Gauss-Seidel method, determine $v_2$ and $v_3$. Perform one iteration only. (ii) If bus voltages converge, determine line flows and losses.
		\begin{figure}[h!]
			\centering
			% NOTE: Replace with image from 2020 (EEE 3211), Page 2.
		%		\includegraphics[width=0.8\textwidth]{path/to/2020_eee3211_q4c.png}
			\caption{Network for 2020, Q.4(c).}
		\end{figure}
	\end{enumerate}
	
	\section{Symmetrical \& Asymmetrical Faults}
	
	\subsection{Year 2021}
	\begin{enumerate}[label=\textbf{Q\arabic*.}, wide, labelindent=0pt, start=5]
		\item 
		\begin{enumerate}[label=\textbf{(\alph*)}]
			\setcounter{enumii}{2} % Start letter at (c)
			\item For the network shown... using "step-by-step $Z_{bus}$ building algorithm", determine the bus impedance matrix. Assume a 3-phase fault occurs at bus 2, find (i) Fault current and (ii) The bus voltages during the fault.
			\begin{figure}[h!]
				\centering
				% NOTE: Replace with image from 2021 (EEE 3211), Page 3.
		%			\includegraphics[width=0.6\textwidth]{path/to/2021_eee3211_q5c.png}
				\caption{Network for 2021, Q.5(c).}
			\end{figure}
		\end{enumerate}
		\item 
		\begin{enumerate}[label=\textbf{(\alph*)}]
			\item When fault occur at the terminal of a generator, prove that, 1LG fault is more severe than 3LG fault if $X_n < \frac{1}{2}(X_1-X_0)$.
			\item A 20MVA, 13.8kV generator has $X''_d = 0.25$ pu, $X_2=0.35$ pu and $X_0=0.10$ pu. ... Determine the subtransient currents and the line to line voltages at fault ... when a 2L fault occurs at the generator terminal.
		\end{enumerate}
	\end{enumerate}
	
	\subsection{Year 2020}
	\begin{enumerate}[label=\textbf{Q.5(b)}, wide, labelindent=0pt]
		\item Draw the positive-, negative-, and zero-sequence network of the following power system.
		\begin{figure}[h!]
			\centering
			% NOTE: Replace with image from 2020 (EEE 3211), Page 2.
		%		\includegraphics[width=0.8\textwidth]{path/to/2020_eee3211_q5b.png}
			\caption{Network for 2020, Q.5(b).}
		\end{figure}
	\end{enumerate}
	
	\subsection{Year 2018}
	\begin{enumerate}[label=\textbf{Q.7(c)}, wide, labelindent=0pt]
		\item Draw the zero sequence equivalent circuits for the following three phase transformer banks.
		\begin{figure}[h!]
			\centering
			% NOTE: Replace with image from 2018 (EEE 3211), Page 2.
		%		\includegraphics[width=\textwidth]{path/to/2018_eee3211_q7c.png}
			\caption{Transformer configurations for 2018, Q.7(c).}
		\end{figure}
	\end{enumerate}
	
	
	\subsection{Year 2016}
	\begin{enumerate}[label=\textbf{Q.5(c)}, wide, labelindent=0pt]
		\item The one line diagram of a simple four-bus power system is shown... A bolted three-phase fault occurs at bus 4. Using Thevenin's theorem obtain the impedances to the point of fault and the fault current in per-unit.
		\begin{figure}[h!]
			\centering
			% NOTE: Replace with image from 2016 (EEE 3211), Page 2.
	%			\includegraphics[width=0.7\textwidth]{path/to/2016_eee3211_q5c.png}
			\caption{Network for 2016, Q.5(c).}
		\end{figure}
	\end{enumerate}
	
	\begin{enumerate}[label=\textbf{Q.4(c)}, wide, labelindent=0pt]
		\item Draw the negative and zero sequence network for the following system.
		\begin{figure}[h!]
			\centering
			% NOTE: Replace with image from 2016 (EEE 3211), Page 1.
		%		\includegraphics[width=0.8\textwidth]{path/to/2016_eee3211_q4c.png}
			\caption{Network for 2016, Q.4(c).}
		\end{figure}
	\end{enumerate}
	\newpage
	%========================================
	% Power System II Topics
	%========================================
	
	\part{Power System II (EEE 4141)}
	
	\section{Power System Stability}
	\subsection{Year 2021}
	\begin{enumerate}[label=\textbf{Q\arabic*.}, wide, labelindent=0pt, start=2]
		\item 
		\begin{enumerate}[label=\textbf{(\alph*)}]
			\setcounter{enumii}{2} % Start sub-enumeration at (c)
			\item A power system is shown below... Determine (i) the power angle equation and (ii) the swing equation during fault.
			\begin{figure}[h!]
				\centering
		%			\includegraphics[width=0.7\textwidth]{path/to/2021_eee4141_q2c.png}
				\caption{Diagram for 2021, Q.2(c).}
			\end{figure}
		\end{enumerate}
	\end{enumerate}
	
	\subsection{Year 2020}
	\begin{enumerate}[label=\textbf{Q\arabic*.}, wide, labelindent=0pt, start=1]
		\item 
		\begin{enumerate}[label=\textbf{(\alph*)}]
			\setcounter{enumii}{1} % Start letter at (b)
			\item The single-line diagram of a power system is shown below...Compute the power angle equation and the swing equation.
			\begin{figure}[h!]
				\centering
		%			\includegraphics[width=0.8\textwidth]{path/to/2020_eee4141_q1b.png}
				\caption{Diagram for 2020, Q.1(b).}
			\end{figure}
		\end{enumerate}
		\item
		\begin{enumerate}[label=\textbf{(\alph*)}]
			\item Derive an expression of the critical clearing angle for the power system as shown...
			\begin{figure}[h!]
				\centering
		%			\includegraphics[width=0.8\textwidth]{path/to/2020_eee4141_q2a.png}
				\caption{Diagram for 2020, Q.2(a).}
			\end{figure}
		\end{enumerate}
	\end{enumerate}
	
	\subsection{Year 2018}
	\begin{enumerate}[label=\textbf{Q.7(c)}, wide, labelindent=0pt]
		\item A 50 Hz synchronous generator...is connected to an infinite bus through a purely reactive circuit as shown... Determine the critical clearing angle and the critical fault clearing time.
		\begin{figure}[h!]
			\centering
		%		\includegraphics[width=0.7\textwidth]{path/to/2018_eee4141_q7c.png}
			\caption{Diagram for 2018, Q.7(c).}
		\end{figure}
	\end{enumerate}
	
	\section{Transmission Lines (Overhead)}
	
	\subsection{Year 2017}
	\begin{enumerate}[label=\textbf{Q.3(c)}, wide, labelindent=0pt]
		\item Find the inductance per phase per km of double circuit 3-phase line shown in the figure.
		\begin{figure}[h!]
			\centering
		%	\includegraphics[width=0.6\textwidth]{path/to/2017_eee4141_q3c.png}
			\caption{Diagram for 2017, Q.3(c).}
		\end{figure}
	\end{enumerate}
	
	\section{Distribution Systems (AC \& DC)}
	\subsection{Year 2021}
	\begin{enumerate}[label=\textbf{Q.7(b)}, wide, labelindent=0pt]
		\item A d.c. 2-wire distributor AB is 500 m long and is fed at both ends at 240 V. The distributor is loaded as shown in given figure... Calculate the point of minimum voltage.
		\begin{figure}[h!]
			\centering
		%		\includegraphics[width=0.8\textwidth]{path/to/2021_eee4141_q7b.png}
			\caption{Diagram for 2021, Q.7(b).}
		\end{figure}
	\end{enumerate}
	
\end{document}