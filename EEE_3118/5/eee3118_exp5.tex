\documentclass[a4paper,12pt]{article}

\usepackage{graphicx} % Required for inserting images
\usepackage{amsmath,amssymb,amsfonts}
\usepackage{subcaption}
% -----------------------
% Package Imports
% -----------------------

% Set page margins
\usepackage[a4paper, top=1in, bottom=0.8in, left=1.1in, right=0.8in]{geometry}

% Use Times New Roman font
\usepackage{times}
\usepackage{array}


% Add page numbering
\pagestyle{plain}
\usepackage{multirow}
% Enable graphics inclusion
\usepackage{graphicx}
\usepackage{float}
% Enable code listings
\usepackage{listings}
\usepackage{xcolor} % For customizing code colors

% Define MATLAB style for listings
\lstdefinestyle{vscode-light}{
	language=Matlab,
	basicstyle=\ttfamily\footnotesize,
	keywordstyle=\color{black},
	commentstyle=\color{gray},
	stringstyle=\color{red},
	numberstyle=\tiny\color{black},
	numbersep=5pt,
	frame=single,
	backgroundcolor=\color{white!10},
	breaklines=true,
	captionpos=b,
	tabsize=4,
	showstringspaces=false,
	numbers=left,  % Enable line numbering on the left
	stepnumber=1,  % Line numbers increment by 1
	numberfirstline=true, % Number the first line
}
\setlength{\parindent}{0pt}
\usepackage{titlesec} % To customize section font size
\titleformat{\section}
{\normalfont\fontsize{14}{16}\bfseries}{\thesection}{1em}{}

\titleformat{\subsection}
{\normalfont\fontsize{14}{16}\bfseries}{\thesubsection}{1em}{}


\begin{document}
	\section{Experiment No. 5}
	
	\section{Experiment Title }
	Observation \& Verification of Fault Signal of an Integrated AM/FM Radio Trainer Base Station.
	\section{Objective}
	
	
	
	The objectives of this lab are as follows:
	\begin{itemize}
		\item To study and verify the basic working principles of an AM/FM radio trainer base station.
		\item	To observe the transmission and reception of Frequency Modulated (FM) signals. 
		\item 	To simulate different types of faults and observe their effects on signal quality and waveform behavior.
	\end{itemize}
	
	\section{Theory}
	
	
	
	The Integrated AM/FM Radio Trainer Base Station is developed to demonstrate the complete process of radio signal reception, demodulation, and fault diagnosis. The trainer receives amplitude modulated (AM) or frequency modulated (FM) radio frequency (RF) signals from external transmitters (KL-93061 for AM and KL-93063 for FM). 
	
	These received signals are first filtered and tuned in the RF section, then converted into an intermediate frequency (IF) for stable and selective amplification. Afterward, the signal passes through a detector circuit that demodulates the signal and retrieves the original audio information from the modulated carrier wave. Finally, the audio amplifier drives the demodulated audio signal to an output device such as a speaker or headphone.
	
	The system also incorporates a fault simulator module, which enables users to analyze and understand the effects of common faults at various stages of the radio circuit.
	\begin{figure}[H]
		\centering
		
		\centering
		\includegraphics[width=0.71\linewidth]{"D:/DOWNLOAD 2024 V2/LATEX FILE/3.1/EEE_3118/5/Images/1"}
		\caption{Integrated AM/FM Radio Trainer Base Station}
		
		
	\end{figure}
	
	
	\subsection*{Modulation and Transmission Process}
	
	The audio signal produced from a source serves as the modulating signal. Before modulation, the audio signal is typically processed using audio conditioning circuits that may include equalizers, compressors, and limiters to enhance sound quality and ensure signal consistency.
	
	Following this, the processed audio signal is fed into a modulator, where it is combined with a high-frequency carrier wave using frequency modulation (FM). The carrier wave is generated by a stable carrier oscillator. The modulated signal, initially weak, is then amplified using a radio frequency (RF) power amplifier to achieve transmission-strength power levels.
	
	The amplified signal is sent to the antenna system, which converts the electrical signal into electromagnetic waves and radiates them into free space. The antenna is carefully tuned to operate efficiently at the desired broadcast frequency.
	
	\subsection*{Reception and Demodulation Process}
	
	The trainer base station is equipped with a complete FM superheterodyne receiver circuit. The reception process begins with the antenna capturing RF signals from the air. Due to their weak nature, these signals are first amplified using an RF amplifier, which also helps suppress unwanted noise.
	
	Next, the signal is passed through a mixer, where it is combined with a locally generated oscillator signal. This process results in the conversion of the input signal to a fixed intermediate frequency (IF), making it easier to process and filter. The IF signal is further amplified to increase clarity and selectivity.
	
	To recover the original audio, the signal undergoes FM demodulation using either a Foster–Seeley discriminator or a phase-locked loop (PLL) circuit. The extracted audio is then amplified by an audio amplifier and finally delivered to a speaker or headphones for listening.
	
	\section*{Fault Detection}
	
	Fault was generated in the RADIO base station by pressing buttons from 1 to 10. The button F4 was pressed before doing every fault detection test because the integrated circuit was in such a manner that the base station normally detects weak signal. So the button is pressed first and then the observations of fault was seen.
	
	
	
	
	
	\section{Required Apparatus}
	
	\begin{enumerate}
		\item Integrated AM/FM Radio Trainer Base Station (LABTECH ERT-AFS)
		\item KL-93061 AM Transmitter Module
		\item KL-93063 FM Transmitter Module
		\item Power Supply
		\item Connecting Wires \& Probes
	\end{enumerate}
	
	
	\section{Experimental Setup}
	\begin{figure}[H]
		\centering
		\begin{subfigure}[t]{0.9\textwidth}
			\centering
			\includegraphics[width=0.81\linewidth]{"D:/DOWNLOAD 2024 V2/LATEX FILE/3.1/EEE_3118/5/Images/1"}
			\caption{Integrated AM/FM Radio Trainer Base Station}
			\vspace{0.1cm}
		\end{subfigure}
		
		\begin{subfigure}[t]{0.9\textwidth}
			\centering
			\includegraphics[width=1\linewidth]{"D:/DOWNLOAD 2024 V2/LATEX FILE/3.1/EEE_3118/5/Images/2"}
			\caption{AM Transmitter (KL-93061)}
		\end{subfigure}
	\end{figure}
	\section{ Experimental Procedure}
	
	\begin{enumerate}
		\item \textbf{Initial Setup:}
		\begin{enumerate}
			\item The Integrated AM/FM Radio Trainer was placed on a flat surface.
			\item The KL-93061 AM and KL-93063 FM transmitter modules were connected to the respective antenna inputs of the trainer.
			\item The trainer and both transmitter modules were powered on.
		\end{enumerate}
		
		\item \textbf{AM Reception Testing:}
		\begin{enumerate}
			\item The trainer was switched to AM mode.
			\item The KL-93061 AM transmitter was enabled to transmit a modulated AM signal.
			\item The AM dial was tuned until the modulated audio signal was received and heard.
			\item The test points were observed using a multimeter or oscilloscope to verify proper signal flow.
		\end{enumerate}
		
		\item \textbf{Fault Simulation:}
		\begin{enumerate}
			\item Predefined faults were introduced using the fault simulator switches on the trainer.
			\item Changes in the audio output and waveform were observed to identify the affected circuit stage.
			\item The faulty block (RF, IF, detector, or amplifier) was diagnosed accordingly.
			
			
			
			\item The trainer and transmitter modules were turned off.
			\item All connecting wires and probes were carefully disconnected.
		\end{enumerate}
	\end{enumerate}
	
	\section{Observation}
	\begin{table}[H]
		\centering
		\caption*{\textbf{Table 5.1:} Radio Receiver Fault Simulation List}
		\begin{tabular}{|
				>{\centering\arraybackslash}p{1cm} |
				>{\centering\arraybackslash}p{2.5cm} |
				>{\centering\arraybackslash}p{2.5cm} |
				>{\centering\arraybackslash}p{3.5cm} |
				>{\centering\arraybackslash}p{2.5cm} |}
			\hline
			\textbf{Fault No:} & 
			\textbf{Defective Circuit} & 
			\textbf{Defective Components} & 
			\textbf{Symptoms} & 
			\textbf{Observation} \\
			\hline
			1 & DC power input & DC power line & System is not working & System is off \\
			\hline
			2 & Stereo decoder & R9 & Stereo indicator does not light & Light is off. \\
			\hline
			3 & Audio output & C2 & No left (L) audio signal & Left audio box is off. \\
			\hline
			4 & Detector \& MPX & IC1 pin6 & Noises and weak signal output & weak \\
			\hline
			5 & FM IF stages & CF2 & Weak FM signal output & weak \\
			\hline
			6 & AM IF stages & L1 & No AM signal output & Not observable. \\
			\hline
			7 & AM RF & R2 & Weak AM signal & Not observable. \\
			\hline
			8 & AM Oscillator & R1 & Weak and noises AM signal & Not observable. \\
			\hline
			9 & FM Tuner & R3 & Can not receive FM signal & Not received. \\
			\hline
			10 & AM AGC & C14 & No AM signal output & Not observable. \\
			\hline
		\end{tabular}
	\end{table}
	\newpage
	
	
	\section{Discussion}
	
	In this experiment, the fault characteristics of the  AM/FM signal using Integrated AM/FM Radio Trainer were analyzed using predefined faults introduced via simulation switches.There were 10 faults that we had check. Faults were examined across key stages such as RF, IF, detector, and audio. Observations were made by monitoring the audio output and measuring signals at test points.\\
	But out of ten,only five faults were successfully observed.	Various symptoms were encountered, including weak signals, distorted or lost audio, and complete system shutdowns.Fault was generated in the RADIO base station by pressing buttons from 1 to 10. The button F4 was pressed before doing every fault detection test because the integrated circuit was in such a manner that the base station normally detects weak signal. So the button is pressed first and then the observations of fault was seen.
	
	
\end{document}