\documentclass{article}
\usepackage{amsmath}
\usepackage{amsfonts}
\usepackage{amssymb}
\usepackage{graphicx} % Required for including images if any, though not for this solution directly.

\begin{document}
	
	\section*{Problem: Root Locus and Stability Analysis}
	
	Consider the following system:
	$$ G(s)H(s) = \frac{k(s+2)}{s(2s+1)(s^2+8s+7)} $$
	
	\subsection*{(i) Draw the complete root locus and determine the stability.}
	
	First, let's express the open-loop transfer function $G(s)H(s)$ in a standard pole-zero form.
	The denominator terms can be factored:
	$2s+1 = 2(s+0.5)$
	$s^2+8s+7 = (s+1)(s+7)$
	
	Substituting these into the expression for $G(s)H(s)$:
	$$ G(s)H(s) = \frac{k(s+2)}{s \cdot 2(s+0.5)(s+1)(s+7)} $$
	$$ G(s)H(s) = \frac{(k/2)(s+2)}{s(s+0.5)(s+1)(s+7)} $$
	Let $K_0 = k/2$. The open-loop transfer function is:
	$$ G(s)H(s) = \frac{K_0(s+2)}{s(s+0.5)(s+1)(s+7)} $$
	
	\subsubsection*{1. Poles and Zeros}
	The poles of the open-loop system (roots of the denominator) are:
	$p_1 = 0$
	$p_2 = -0.5$
	$p_3 = -1$
	$p_4 = -7$
	Number of poles, $N_p = 4$.
	
	The zeros of the open-loop system (roots of the numerator) are:
	$z_1 = -2$
	Number of zeros, $N_z = 1$.
	
	\subsubsection*{2. Number of Root Locus Branches}
	The number of root locus branches is equal to the number of poles.
	Number of branches $= N_p = 4$.
	
	\subsubsection*{3. Real Axis Segments}
	A point on the real axis is part of the root locus if the total number of poles and zeros to its right is odd.
	\begin{itemize}
		\item To the right of $s=0$: Number of poles/zeros is 0 (even). $\implies$ NOT on RL.
		\item Between $s=-0.5$ and $s=0$: Number of poles/zeros is 1 (pole at $s=0$) (odd). $\implies$ \textbf{ON RL}.
		\item Between $s=-1$ and $s=-0.5$: Number of poles/zeros is 2 (poles at $s=0, -0.5$) (even). $\implies$ NOT on RL.
		\item Between $s=-2$ and $s=-1$: Number of poles/zeros is 3 (poles at $s=0, -0.5, -1$) (odd). $\implies$ \textbf{ON RL}.
		\item Between $s=-7$ and $s=-2$: Number of poles/zeros is 4 (zero at $s=-2$, poles at $s=-1, -0.5, 0$) (even). $\implies$ NOT on RL.
		\item To the left of $s=-7$: Number of poles/zeros is 5 (zero at $s=-2$, poles at $s=-1, -0.5, 0, -7$) (odd). $\implies$ \textbf{ON RL}.
	\end{itemize}
	So, the root locus exists on the real axis in the intervals: $(-0.5, 0)$, $(-2, -1)$, and $(-\infty, -7)$.
	
	\subsubsection*{4. Asymptotes}
	The number of asymptotes is $N_p - N_z = 4 - 1 = 3$.
	\begin{itemize}
		\item \textbf{Angles of Asymptotes ($\phi_a$):}
		$$ \phi_a = \frac{(2q+1)180^\circ}{N_p - N_z} \quad \text{for } q = 0, 1, \dots, N_p-N_z-1 $$
		For $q=0: \phi_a = \frac{1 \cdot 180^\circ}{3} = 60^\circ$
		For $q=1: \phi_a = \frac{3 \cdot 180^\circ}{3} = 180^\circ$
		For $q=2: \phi_a = \frac{5 \cdot 180^\circ}{3} = 300^\circ \equiv -60^\circ$
		The angles are $60^\circ, 180^\circ, -60^\circ$.
		
		\item \textbf{Centroid of Asymptotes ($\sigma_a$):}
		$$ \sigma_a = \frac{\sum (\text{real parts of poles}) - \sum (\text{real parts of zeros})}{N_p - N_z} $$
		$$ \sigma_a = \frac{(0 + (-0.5) + (-1) + (-7)) - (-2)}{4 - 1} = \frac{-8.5 + 2}{3} = \frac{-6.5}{3} \approx -2.167 $$
	\end{itemize}
	
	\subsubsection*{5. Breakaway/Break-in Points}
	The characteristic equation is $1 + G(s)H(s) = 0$.
	$$ 1 + \frac{K_0(s+2)}{s(s+0.5)(s+1)(s+7)} = 0 $$
	$$ s(s+0.5)(s+1)(s+7) + K_0(s+2) = 0 $$
	$$ K_0 = -\frac{s(s+0.5)(s+1)(s+7)}{s+2} $$
	$$ K_0 = -\frac{s(s^2+1.5s+0.5)(s+7)}{s+2} $$
	$$ K_0 = -\frac{s(s^3+7s^2+1.5s^2+10.5s+0.5s+3.5)}{s+2} $$
	$$ K_0 = -\frac{s^4+8.5s^3+11s^2+3.5s}{s+2} $$
	To find breakaway/break-in points, we set $\frac{dK_0}{ds} = 0$.
	Let $N(s) = s^4+8.5s^3+11s^2+3.5s$ and $D(s) = s+2$.
	Then $\frac{dK_0}{ds} = -\frac{N'(s)D(s) - N(s)D'(s)}{[D(s)]^2}$. Setting it to zero implies $N'(s)D(s) - N(s)D'(s) = 0$.
	$N'(s) = 4s^3 + 25.5s^2 + 22s + 3.5$
	$D'(s) = 1$
	So, $(4s^3 + 25.5s^2 + 22s + 3.5)(s+2) - (s^4 + 8.5s^3 + 11s^2 + 3.5s)(1) = 0$
	Expanding and simplifying, we get:
	$3s^4 + 25s^3 + 62s^2 + 44s + 7 = 0$
	Solving this quartic equation numerically, the real roots are approximately:
	$s_1 \approx -0.21$
	$s_2 \approx -0.91$
	(Other roots are not on the valid real axis segments for $K_0 > 0$).
	
	We check if these points lie on the valid root locus segments for $K_0 > 0$:
	\begin{itemize}
		\item At $s \approx -0.21$: This point is between $s=0$ and $s=-0.5$, which is a valid segment.
		Substituting $s = -0.21$ into the $K_0$ equation:
		$K_0 \approx -\frac{(-0.21)^4 + 8.5(-0.21)^3 + 11(-0.21)^2 + 3.5(-0.21)}{-0.21+2} \approx 0.18$.
		Since $K_0 > 0$, $s \approx -0.21$ is a valid \textbf{breakaway point}.
		\item At $s \approx -0.91$: This point is between $s=-1$ and $s=-2$, which is a valid segment.
		Substituting $s = -0.91$ into the $K_0$ equation:
		$K_0 \approx -\frac{(-0.91)^4 + 8.5(-0.91)^3 + 11(-0.91)^2 + 3.5(-0.91)}{-0.91+2} \approx -0.16$.
		Since $K_0 < 0$, $s \approx -0.91$ is NOT a valid breakaway/break-in point for positive $K_0$.
	\end{itemize}
	Therefore, there is one valid breakaway point at $s \approx -0.21$.
	
	\subsubsection*{6. Intersection with Imaginary Axis (J$\omega$-axis crossover)}
	We use the Routh-Hurwitz criterion on the characteristic equation:
	$s^4 + 8.5s^3 + 11s^2 + 3.5s + K_0(s+2) = 0$
	$s^4 + 8.5s^3 + 11s^2 + (3.5+K_0)s + 2K_0 = 0$
	
	The Routh Array is:


		
		The Routh Array is:
		\begin{equation*}
			\begin{array}{|c|c|c|c|}
				\hline
				s^4 & 1 & 11 & 2K_0 \\
				\hline
				s^3 & 8.5 & (3.5+K_0) & 0 \\
				\hline
				s^2 & b_1 & b_2 & 0 \\
				\hline
				s^1 & c_1 & 0 & 0 \\
				\hline
				s^0 & d_1 & 0 & 0 \\
				\hline
			\end{array}
		\end{equation*}
		
		Where the elements are calculated as:
		% ... (rest of your calculations)
		

	
	$b_1 = \frac{8.5 \cdot 11 - 1 \cdot (3.5+K_0)}{8.5} = \frac{93.5 - 3.5 - K_0}{8.5} = \frac{90 - K_0}{8.5}$
	$b_2 = \frac{8.5 \cdot 2K_0 - 1 \cdot 0}{8.5} = 2K_0$
	
	For the system to be stable, all elements in the first column must be positive:
	\begin{itemize}
		\item $8.5 > 0$ (satisfied)
		\item $\frac{90 - K_0}{8.5} > 0 \implies 90 - K_0 > 0 \implies K_0 < 90$
		\item $2K_0 > 0 \implies K_0 > 0$
		\item For $c_1$: $c_1 = \frac{b_1 (3.5+K_0) - 8.5 b_2}{b_1}$. For $c_1 > 0$, the numerator must be positive:
		$\frac{90 - K_0}{8.5} (3.5+K_0) - 8.5 (2K_0) > 0$
		$(90 - K_0)(3.5+K_0) - 17K_0 \cdot 8.5 > 0$
		$315 + 90K_0 - 3.5K_0 - K_0^2 - 144.5K_0 > 0$
		$-K_0^2 - 58K_0 + 315 > 0$
		$K_0^2 + 58K_0 - 315 < 0$
	\end{itemize}
	To find the roots of $K_0^2 + 58K_0 - 315 = 0$:
	$$ K_0 = \frac{-58 \pm \sqrt{58^2 - 4(1)(-315)}}{2} = \frac{-58 \pm \sqrt{3364 + 1260}}{2} = \frac{-58 \pm \sqrt{4624}}{2} = \frac{-58 \pm 68}{2} $$
	$K_{0,1} = \frac{-58 + 68}{2} = 5$
	$K_{0,2} = \frac{-58 - 68}{2} = -63$
	So, $K_0^2 + 58K_0 - 315 < 0$ implies $-63 < K_0 < 5$.
	
	Combining all conditions for stability ($K_0 > 0$, $K_0 < 90$, and $-63 < K_0 < 5$), the system is stable for $0 < K_0 < 5$.
	
	The system just oscillates when $c_1 = 0$, which occurs when $K_0 = 5$.
	At $K_0 = 5$, the auxiliary equation is formed from the $s^2$ row:
	$\frac{90 - 5}{8.5} s^2 + 2(5) = 0$
	$\frac{85}{8.5} s^2 + 10 = 0$
	$10s^2 + 10 = 0$
	$s^2 = -1 \implies s = \pm j1$
	These are the points where the root locus crosses the imaginary axis.
	
	\subsubsection*{7. Angle of Departure/Arrival}
	Not applicable as there are no complex poles or zeros.
	
	\subsubsection*{8. Root Locus Sketch}
	\begin{itemize}
		\item Poles at $0, -0.5, -1, -7$. Zero at $-2$.
		\item Real Axis Segments: $(-0.5, 0)$, $(-2, -1)$, and $(-\infty, -7)$.
		\item Centroid $\approx -2.167$. Asymptote angles: $60^\circ, 180^\circ, -60^\circ$.
		\item Breakaway point at $s \approx -0.21$.
		\item J$\omega$-axis crossover at $s = \pm j1$ when $K_0 = 5$ (which means $k=10$).
	\end{itemize}
	
	\textbf{Description of Root Locus Branches:}
	\begin{enumerate}
		\item Two branches start from the poles at $s=0$ and $s=-0.5$. They move towards each other along the real axis, meet at the breakaway point $s \approx -0.21$, and then break away into the complex plane.
		\item These two complex conjugate branches move towards the imaginary axis. They cross the imaginary axis at $s = \pm j1$ when $K_0=5$ (or $k=10$). For $K_0 > 5$, these branches enter the right-half plane. As $K_0 \to \infty$, they approach the asymptotes at $\pm 60^\circ$.
		\item One branch starts from the pole at $s=-1$ and moves along the real axis to the left, terminating at the finite zero at $s=-2$. This forms a closed segment between $s=-1$ and $s=-2$.
		\item The fourth branch starts from the pole at $s=-7$ and moves along the real axis to the left, approaching negative infinity. This branch approaches the $180^\circ$ asymptote.
	\end{enumerate}
	
	\begin{figure}[h!]
		\centering
	%	\includegraphics[width=0.8\textwidth]{root_locus_sketch.png} % Placeholder for the sketch. In a real document, you'd insert the plot generated by MATLAB.
		\caption{Sketch of the Root Locus for $G(s)H(s) = \frac{K_0(s+2)}{s(s+0.5)(s+1)(s+7)}$}
		\label{fig:root_locus}
	\end{figure}
	\textit{Note: The image above is a placeholder. In a practical solution, the actual plot generated by software like MATLAB would be inserted here.}
	
	\subsubsection*{Stability Determination}
	Based on the Routh-Hurwitz criterion and the root locus plot:
	\begin{itemize}
		\item The system is \textbf{stable} for $0 < K_0 < 5$.
		Since $K_0 = k/2$, this corresponds to $0 < k/2 < 5$, or $0 < k < 10$.
		For these values of $k$, all closed-loop poles lie in the left-half of the s-plane.
		\item The system is \textbf{marginally stable} (just oscillates) for $K_0 = 5$ (i.e., $k=10$).
		At this value, two closed-loop poles are located on the imaginary axis at $s = \pm j1$.
		\item The system is \textbf{unstable} for $K_0 > 5$ (i.e., $k > 10$).
		For these values of $k$, two closed-loop poles move into the right-half of the s-plane, causing the system output to grow unbounded.
	\end{itemize}
	
\end{document}