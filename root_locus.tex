\documentclass{article}
\usepackage{amsmath}
\usepackage{graphicx}
\usepackage{tikz}
\usetikzlibrary{arrows.meta,calc,decorations.markings}

\begin{document}
	
	\section*{Root Locus Analysis for \( G(s)H(s) = \frac{k(s+2)}{s(2s+1)(s^2+8s+7)} \)}
	
	\subsection*{Step 1: Factorize the Denominator}
	The denominator is \( s(2s+1)(s^2+8s+7) \). We factorize the quadratic term:
	\[ s^2 + 8s + 7 = (s+1)(s+7) \]
	Thus, the transfer function becomes:
	\[ G(s)H(s) = \frac{k(s+2)}{s(2s+1)(s+1)(s+7)} \]
	The constant factor 2 can be absorbed into \(k\):
	\[ G(s)H(s) = \frac{K(s+2)}{s(s+0.5)(s+1)(s+7)}, \quad \text{where } K = \frac{k}{2} \]
	
	\subsection*{Step 2: Identify Poles and Zeros}
	\begin{itemize}
		\item \textbf{Poles} (denominator roots): \(s = 0\), \(s = -0.5\), \(s = -1\), \(s = -7\) (total = 4 poles)
		\item \textbf{Zero} (numerator root): \(s = -2\) (total = 1 zero)
	\end{itemize}
	
	\subsection*{Step 3: Real-Axis Segments}
	The root locus exists on the real axis where the number of poles and zeros to the right is odd. The intervals are:
	\begin{itemize}
		\item \((-\infty, -7]\): 4 poles + 1 zero to right → odd → \textbf{included}
		\item \([-2, -1]\): 2 poles to right → even → \textbf{excluded}
		\item \([-0.5, 0]\): 1 pole to right → odd → \textbf{included}
	\end{itemize}
	
	\subsection*{Step 4: Asymptotes}
	Number of asymptotes = \(n - m = 4 - 1 = 3\):
	\begin{itemize}
		\item \textbf{Angles}:
		\[ \theta = \frac{(2q+1)180^\circ}{3} \quad \text{for} \quad q=0,1,2 \]
		\[ \theta = 60^\circ, 180^\circ, 300^\circ \]
		\item \textbf{Centroid}:
		\[ \sigma = \frac{0 - 0.5 - 1 - 7 - (-2)}{3} = \frac{-6.5}{3} \approx -2.167 \]
	\end{itemize}
	
	\subsection*{Step 5: Breakaway/Break-in Points}
	Solve \(\frac{dK}{ds} = 0\) where:
	\[ K = -\frac{s(s+0.5)(s+1)(s+7)}{s+2} \]
	Numerical solution yields:
	\begin{itemize}
		\item Breakaway point in \([-0.5, 0]\) at \(s \approx -0.223\) (\(K \approx 0.366\))
		\item No valid break-in points on other segments
	\end{itemize}
	
	\subsection*{Step 6: Imaginary Axis Crossings}
	Characteristic equation:
	\[ 1 + G(s)H(s) = 0 \implies s(s+0.5)(s+1)(s+7) + K(s+2) = 0 \]
	\[ s^4 + 8.5s^3 + 15.5s^2 + 3.5s + Ks + 2K = 0 \]
	Using Routh-Hurwitz criterion, we find:
	\begin{itemize}
		\item Stable for \(0 < K < 10\)
		\item Marginally stable at \(K = 10\) with poles at \(s = \pm j\)
		\item Unstable for \(K > 10\)
	\end{itemize}
	
	\subsection*{Root Locus Sketch}
	\begin{center}
		\begin{tikzpicture}[scale=1.2,>=latex]
			% Axes
			\draw[->] (-8,0) -- (2,0) node[below]{$\sigma$};
			\draw[->] (0,-4) -- (0,4) node[left]{$j\omega$};
			
			% Poles and zeros
			\filldraw (0,0) circle (2pt) node[below right]{$0$};
			\filldraw (-0.5,0) circle (2pt) node[below]{$-0.5$};
			\filldraw (-1,0) circle (2pt) node[below]{$-1$};
			\filldraw (-7,0) circle (2pt) node[below]{$-7$};
			\draw[fill=white] (-2,0) circle (2pt) node[above]{$-2$};
			
			% Asymptotes
			\draw[dashed] (-2.167,0) -- ++(60:4);
			\draw[dashed] (-2.167,0) -- ++(180:4);
			\draw[dashed] (-2.167,0) -- ++(300:4);
			
			% Root locus branches
			\draw[thick,red,->] (-7,0) -- (-8,0);
			\draw[thick,red,->] (-1,0) -- (-2,0);
			\draw[thick,red] (0,0) to[out=90,in=180] (-0.223,0.5);
			\draw[thick,red] (-0.5,0) to[out=90,in=0] (-0.223,0.5);
			\draw[thick,red,->] (-0.223,0.5) -- ++(60:1.5);
			\draw[thick,red,->] (-0.223,-0.5) -- ++(300:1.5);
			
			% Imaginary axis crossing
			\filldraw (0,1) circle (2pt) node[above right]{$j1$};
			\filldraw (0,-1) circle (2pt) node[below right]{$-j1$};
		\end{tikzpicture}
	\end{center}
	
	\subsection*{Stability Conclusion}
	\begin{itemize}
		\item \textbf{Stable} for \(0 < K < 10\) (all poles in LHP)
		\item \textbf{Marginally stable} at \(K = 10\) (poles at \(s = \pm j\))
		\item \textbf{Unstable} for \(K > 10\) (poles in RHP)
	\end{itemize}
	
\end{document}